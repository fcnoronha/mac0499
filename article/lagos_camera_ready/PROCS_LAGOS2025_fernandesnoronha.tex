% Template for Elsevier CRC journal article
% version 1.2 dated 09 May 2011

% This file (c) 2009-2011 Elsevier Ltd.  Modifications may be freely made,
% provided the edited file is saved under a different name

% This file contains modifications for Procedia Computer Science

% Changes since version 1.1
% - added "procedia" option compliant with ecrc.sty version 1.2a
%   (makes the layout approximately the same as the Word CRC template)
% - added example for generating copyright line in abstract

%-----------------------------------------------------------------------------------

%% This template uses the elsarticle.cls document class and the extension package ecrc.sty
%% For full documentation on usage of elsarticle.cls, consult the documentation "elsdoc.pdf"
%% Further resources available at http://www.elsevier.com/latex

%-----------------------------------------------------------------------------------

%%%%%%%%%%%%%%%%%%%%%%%%%%%%%%%%%%%%%%%%%%%%%%%%%%%%%%%%%%%%%%
%%%%%%%%%%%%%%%%%%%%%%%%%%%%%%%%%%%%%%%%%%%%%%%%%%%%%%%%%%%%%%
%%                                                          %%
%% Important note on usage                                  %%
%% -----------------------                                  %%
%% This file should normally be compiled with PDFLaTeX      %%
%% Using standard LaTeX should work but may produce clashes %%
%%                                                          %%
%%%%%%%%%%%%%%%%%%%%%%%%%%%%%%%%%%%%%%%%%%%%%%%%%%%%%%%%%%%%%%
%%%%%%%%%%%%%%%%%%%%%%%%%%%%%%%%%%%%%%%%%%%%%%%%%%%%%%%%%%%%%%

%% The '3p' and 'times' class options of elsarticle are used for Elsevier CRC
%% The 'procedia' option causes ecrc to approximate to the Word template
\documentclass[3p,times,procedia]{elsarticle}
\flushbottom

%% The `ecrc' package must be called to make the CRC functionality available
\usepackage{ecrc}
\usepackage[bookmarks=false]{hyperref}
    \hypersetup{colorlinks,
      linkcolor=blue,
      citecolor=blue,
      urlcolor=blue}
%\usepackage{amsmath}

\usepackage{amsmath,amssymb,amsthm}    
% \usepackage{amsaddr}
% \usepackage{tikz}
\usepackage{algpseudocode}
\usepackage{algorithm}

\usepackage{standalone}
\usepackage{mathtools}
\DeclarePairedDelimiter\ceil{\lceil}{\rceil}
\DeclarePairedDelimiter\floor{\lfloor}{\rfloor}

\usepackage{letltxmacro}
\LetLtxMacro{\oldsqrt}{\sqrt}
\renewcommand{\sqrt}[2][\mkern8mu]{\mkern-1mu\mathop{\oldsqrt[#1]{#2}}}

\newcommand{\NP}{\mbox{NP}}
\newcommand{\PP}{\mbox{P}}
\newcommand{\eps}{\epsilon}
\newcommand{\calP}{\mathcal{P}}
\newcommand{\calT}{\mathcal{T}}
\newcommand{\Oh}{\mathrm{O}}

\newtheorem{theorem}              {Theorem}[section]
\newtheorem{lemma}     	[theorem] {Lemma}        
\newtheorem{conjecture}	[theorem] {Conjecture}   
\newtheorem{property}  	[theorem] {Property}   
\newtheorem{definition}	[theorem] {Definition}   
\newtheorem{proposition}[theorem] {Proposition}   
\newtheorem{corollary}	[theorem] {Corollary}
\newtheorem{fact}	[theorem] {Fact}     
\newtheorem{claim}	[theorem] {Claim}  


%% The ecrc package defines commands needed for running heads and logos.
%% For running heads, you can set the journal name, the volume, the starting page and the authors

%% set the volume if you know. Otherwise `00'
\volume{00}

%% set the starting page if not 1
\firstpage{1}

%% Give the name of the journal
\journalname{Procedia Computer Science}

%% Give the author list to appear in the running head
%% Example \runauth{C.V. Radhakrishnan et al.}
\runauth{C. G. Fernandes, F.C. Noronha}

%% The choice of journal logo is determined by the \jid and \jnltitlelogo commands.
%% A user-supplied logo with the name <\jid>logo.pdf will be inserted if present.
%% e.g. if \jid{yspmi} the system will look for a file yspmilogo.pdf
%% Otherwise the content of \jnltitlelogo will be set between horizontal lines as a default logo

%% Give the abbreviation of the Journal.
\jid{procs}

%% Give a short journal name for the dummy logo (if needed)
%\jnltitlelogo{Computer Science}

%% Hereafter the template follows `elsarticle'.
%% For more details see the existing template files elsarticle-template-harv.tex and elsarticle-template-num.tex.

%% Elsevier CRC generally uses a numbered reference style
%% For this, the conventions of elsarticle-template-num.tex should be followed (included below)
%% If using BibTeX, use the style file elsarticle-num.bst

%% End of ecrc-specific commands
%%%%%%%%%%%%%%%%%%%%%%%%%%%%%%%%%%%%%%%%%%%%%%%%%%%%%%%%%%%%%%%%%%%%%%%%%%

%% The amssymb package provides various useful mathematical symbols

\usepackage{amssymb}
%% The amsthm package provides extended theorem environments
%% \usepackage{amsthm}

%% The lineno packages adds line numbers. Start line numbering with
%% \begin{linenumbers}, end it with \end{linenumbers}. Or switch it on
%% for the whole article with \linenumbers after \end{frontmatter}.
%% \usepackage{lineno}

%% natbib.sty is loaded by default. However, natbib options can be
%% provided with \biboptions{...} command. Following options are
%% valid:

%%   round  -  round parentheses are used (default)
%%   square -  square brackets are used   [option]
%%   curly  -  curly braces are used      {option}
%%   angle  -  angle brackets are used    <option>
%%   semicolon  -  multiple citations separated by semi-colon
%%   colon  - same as semicolon, an earlier confusion
%%   comma  -  separated by comma
%%   numbers-  selects numerical citations
%%   super  -  numerical citations as superscripts
%%   sort   -  sorts multiple citations according to order in ref. list
%%   sort&compress   -  like sort, but also compresses numerical citations
%%   compress - compresses without sorting
%%
%% \biboptions{authoryear}

% \biboptions{}

% if you have landscape tables
\usepackage[figuresright]{rotating}
%\usepackage{harvard}
% put your own definitions here:x
%   \newcommand{\cZ}{\cal{Z}}
%   \newtheorem{def}{Definition}[section]
%   ...

% add words to TeX's hyphenation exception list
%\hyphenation{author another created financial paper re-commend-ed Post-Script}

% declarations for front matter


\begin{document}
\begin{frontmatter}

%% Title, authors and addresses

%% use the tnoteref command within \title for footnotes;
%% use the tnotetext command for the associated footnote;
%% use the fnref command within \author or \address for footnotes;
%% use the fntext command for the associated footnote;
%% use the corref command within \author for corresponding author footnotes;
%% use the cortext command for the associated footnote;
%% use the ead command for the email address,
%% and the form \ead[url] for the home page:
%%
%% \title{Title\tnoteref{label1}}
%% \tnotetext[label1]{}
%% \author{Name\corref{cor1}\fnref{label2}}
%% \ead{email address}
%% \ead[url]{home page}
%% \fntext[label2]{}
%% \cortext[cor1]{}
%% \address{Address\fnref{label3}}
%% \fntext[label3]{}

\dochead{XIII Latin American Algorithms, Graphs, and Optimization Symposium (LAGOS 2025)}%%%
%% Use \dochead if there is an article header, e.g. \dochead{Short communication}
%% \dochead can also be used to include a conference title, if directed by the editors
%% e.g. \dochead{17th International Conference on Dynamical Processes in Excited States of Solids}

\title{How to go from partial to full retroactivity in detail}

%% use optional labels to link authors explicitly to addresses:
%% \author[label1,label2]{<author name>}
%% \address[label1]{<address>}
%% \address[label2]{<address>}

\author[a]{Cristina G. Fernandes} 
\ead{cris@ime.usp.br}
\author[a]{Felipe C. Noronha}
\ead{felipe.castro.noronha@hotmail.com}
% \author[a,b]{Third Author\corref{cor1}}

\address[a]{Department of Computer Science, University of São Paulo, Brazil}

\begin{abstract}
  The concept of retroactivity in data structures was introduced by Demaine, Iacono, 
  and Langerman.  In their original paper on retroactivity, they described a way to 
  transform a partially retroactive data structure into a fully retroactive one.
  Their description is focused on space savings, which require the use of a
  persistent version of the data structure involved.  We focus on the case in 
  which one does not have or does not want to use a persistent data structure. 
  We describe and analyze in detail how to implement their transformation in this case.
  As a secondary contribution, using our strategy, we implemented a (halfway) 
  retroactive data structure for the incremental minimum spanning forest (MSF) problem, 
  that we make available. 
\end{abstract}

\begin{keyword}
  Retroactive data structures \sep incremental minimum spanning forest \sep link-cut trees
\end{keyword}


% \cortext[cor1]{Corresponding author}
\end{frontmatter}

%\correspondingauthor[*]{Corresponding author. Tel.: +0-000-000-0000 ; fax: +0-000-000-0000.}

%%
%% Start line numbering here if you want
%%
% \linenumbers
\vspace*{-6pt}
%% main text

%\enlargethispage{-7mm}
\section{Introduction}
\label{Introduction}

Problems in dynamic graphs have many applications, as they can be used to model a variety 
of real situations where the graph models a network of sorts that is changing over time. 
A subclass of these problems that are already interesting and challenging are the so-called 
incremental problems, in which the considered graph is growing with time, through the 
addition of edges.  

The \emph{Minimum Spanning Tree} problem consists of, given a connected 
graph $G$ with costs on its edges, finding a spanning tree of~$G$ with minimum cost. 
To describe the incremental version of this well-known problem, 
we consider a generalization on graphs that are not necessarily connected. 
The \emph{Minimum Spanning Forest (MSF)} problem consists of, given a graph~$G$ with 
costs on its edges, finding a maximal spanning forest of~$G$ with minimum cost. 

The \emph{incremental MSF} is the problem of keeping track of an MSF in a graph on 
$n$ vertices that is changing through the addition of new edges with specified costs.  
We may assume the initial graph is empty. 
Frederickson~\cite{Frederickson1983} described how to solve this problem efficiently 
using link-cut trees, and addressed the more general dynamic MSF,
that also allows deletion of edges.  The cost per update of his method is $\Oh(\sqrt{m})$, 
where $m$ is the number of edges in the graph at the moment of the update.

The concept of retroactivity in data structures was introduced by Demaine, Iacono, 
and Langerman~\cite{DemaineIL2007}.  Its applications include practical situations 
where the involved data might be manipulated in imperfect ways, and once in a while 
there is a need to correct some erroneous operation done, or to perform some operation 
that was forgotten.

A data structure usually gives support to updates and queries.  
Generally, the order in which the updates are performed interferes with the state 
of the data structure. 
Consider a data structure that starts empty, and suffers a sequence of updates, 
each with a time stamp that registers the time it occurred. 
The goal of retroactivity is to allow one to efficiently manipulate this update sequence, 
and to answer queries not only on the current state of the data structure, 
but also on the state of the data structure at any time~$t$, that is, 
the state in which the data structure would be if we applied only the updates 
in the sequence with time stamp at most~$t$.
Specifically, in the context of retroactivity, one wants to be able to insert into 
the sequence an update with a time stamp $t$, possibly indicating a time in the past, 
and to remove some update from the sequence, given its time stamp.
We assume the time stamps are all distinct.  Moreover, given a time~$t$, 
one would like to answer queries on the state of the data structure at time~$t$. 

If one can efficiently answer only queries on the current state of the data structure, 
but not on its state at an arbitrary time $t$, the data structure is said to be 
\emph{partially retroactive}.  In the literature, retroactivity is sometimes used 
to refer to all variants of retroactivity, and the expression \emph{fully retroactive}
is then used to refer to an implementation that provides the complete set of retroactive
operations: insertion and removal of updates at any time, as well as answering queries 
at any time.  

Demaine, Iacono, and Langerman~\cite{DemaineIL2007} described fully 
retroactive versions of queues, doubly ended queues, priority queue, union-find, 
and also a more efficient partially retroactive priority queue.
They also described how to transform a partially retroactive data structure into 
a fully retroactive one with an $\Oh(\sqrt{m})$ slowdown per update operation, 
where $m$ is the length of the update sequence. 
In general, assuming that certain known conjectures in complexity theory hold, 
this slowdown is essentially tight~\cite{ChenDGWXY2018}.
The transformation is done, at first, by keeping $\Oh(\sqrt{m})$ versions of the 
partially retroactive data structure. 
Under a condition that assures that there is an efficient persistent version 
of the data structure involved, they show how to achieve a more effective space
consumption without losing in the time consumption. 
(A persistent data structure is a data structure that always preserves the 
previous version of itself when it is modified~\cite{DriscollSST1989}.)
Years later, Demaine et al.~\cite{DemaineKLSY2015} provided a transformation 
from \emph{time-fusible} partially retroactive data structures into fully
retroactive ones, with a logarithmic time slowdown per operation and applied
this transformation to obtain a more efficient fully retroactive priority queue.

It is known that a data structure used to solve a dynamic problem, such as the 
dynamic MSF problem, can be used as a partially retroactive solution for the problem.  
For instance, an efficient data structure for the dynamic MSF problem works as an 
efficient partially retroactive MSF solution: the insertions and removals of edges 
of the graph are the updates, and the query is the cost of an MSF in the current graph.  
For partial retroactivity, addition or removal of edges at any time $t$ can be made 
in the present version, and as addition and removal are the inverse of each other, 
one achieves partial retroactivity.  
There are efficient implementations for dynamic MSF~\cite{HolmLT2001,HolmRWN2015}, 
that assure $\Oh(\lg^4 n)$ time amortized per operation, for simple graphs on $n$ vertices. 
So the same bound per operation holds for the partially retroactive MSF problem. 

Recently, Henzinger and Wu~\cite{HenzingerW2021} presented lower bounds for the 
time per operation of a fully retroactive data structure for the MSF problem and 
for connectivity, under the OMv conjecture~\cite{HenzingerKNS2015}.  The lower 
bounds are in terms of the number $n$ of vertices of the graph: for any $\eps>0$,
there is no fully retroactive solution that takes $\Oh(n^{1-\eps})$ time per 
operation for these problems.  The authors also presented a fully retroactive
data structure for connectivity, maximum degree, and MSF in $\tilde{\Oh}(n)$ 
per operation.  

The study of de Andrade Junior and Seabra~\cite{deAndradeJrS2022} about retroactivity 
addresses the incremental MSF problem.  In the incremental MSF problem, the only update
supported is the addition of edges.  So the update sequence, in this case, consists 
of a series of edge additions.  To support full retroactivity, one would have to give 
support to the insertion of new edge additions at any time, and also to the removal 
of an edge addition that occurred at some given time $t$.  Their implementation 
gives support to edge addition at any time~$t$ and answers queries at any time~$t$.  
It does not allow for the removal of an edge addition from the update sequence, 
so we refer to this as a semi-retroactive incremental MSF solution.
(Roditty and Zwick~\cite{RodittyZ2016}, studying strong connectivity, considered
yet another version of retroactivity that was called \emph{incremental}, 
where one is allowed to add an edge only at the present time, not in the past, 
but can remove from the update sequence any edge addition, given its time stamp.) 

The implementation of de Andrade Junior and Seabra is inspired on the
aforementioned technique of Demaine, Iacono, and Langerman~\cite[Theorem~5]{DemaineIL2007}
for transforming partially retroactive data structures into fully retroactive ones.
This technique uses the idea of square-root decomposition, that breaks a time line 
of length~$m$ into $\sqrt{m}$ checkpoints, keeping the state of the data structure 
at these $\sqrt{m}$ checkpoints, as well as the whole sequence of updates.  
To answer queries at an arbitrary time $t$, 
it computes what is the checkpoint previous to~$t$, as close as possible to $t$, 
and then it temporarily applies, to the data structure of this checkpoint, 
the updates between the checkpoint and $t$, to be able to answer the query properly.  
After answering the query, 
it rolls back these updates to recover the checkpoint state of the data structure.  
For the purpose of their experimental study, de Andrade Junior and Seabra assumed
the length $m$ of the time line was known from the start, and that the updates
had time stamps from 1 to $m$, so they do not ever rebuild the data structure.
Also, as Frederickson~\cite{Frederickson1983}, 
they used link-cut trees as the data structure for each checkpoint.  
This leads to an amortized query and edge addition time of $\Oh(\sqrt{m}\lg n)$.  
The space used by their implementation is $\Theta(m\sqrt{m})$ 
because they used a collection of $\Theta(\sqrt{m})$ independent link-cut trees. 
Our initial goal in this work was to improve de Andrade Junior and Seabra's 
implementation, allowing the length of the update sequence for the incremental 
MSF to grow arbitrarily, without increasing its running time or space consumption.

Theorem~5 in~\cite{DemaineIL2007} has two parts.  In the first part, the authors
are concerned with the running time, and describe the above mentioned technique.
At this point, they do not give details on how to keep the~$\sqrt{m}$ checkpoints 
and the data structures so that, after many insertions and removals of operations, 
at most~$(3/2)\sqrt{m}$ operations occur between consecutive checkpoints, with~$m$ 
changing and all. The second part of the proof is concerned with saving space, and 
describes how to rebuild the data structure from time to time to keep this invariant.  
In order to save space, under the same time consumption, their technique uses 
a persistent version of the data structure in this rebuilding process.  
There are sophisticated functional implementations of link-cut trees described in 
the literature~\cite{DemaineLP2008}, based on the use of the so called \emph{fingers}.  
But we were not focused on the space savings, so we concentrated on looking for 
a simple way to keep the $\sqrt{m}$ checkpoints and the link-cut trees so that, 
after many insertions and removals of operations, at most $(3/2)\sqrt{m}$ 
operations occur between consecutive checkpoints, with~$m$ changing and all.
We did not find an efficient way along the lines of rebalancing of B-tree nodes, 
so we considered the rebuilding process.  But rebuilding from scratch a whole 
new collection of link-cut trees from time to time would extrapolate the 
amortized time by operation.  So, we came up with a simple way to use the 
previous version of the collection of independent link-cut trees to build 
its new version.  

Our main contribution is that our strategy can be applied in general, 
to transform any partially retroactive data structure into a fully retroactive 
one, when one is not willing to use or does not have a persistent version of 
the involved data structure to apply the space saving strategy from Theorem~5 
in~\cite{DemaineIL2007}.  It achieves the same slowdown in updates and queries 
that their technique.  We applied our strategy to obtain the improvement we 
wanted on de Andrade Junior and Seabra's implementation.  So, a secondary 
contribution is an implementation, whose code we made available, of a 
semi-retroactive version for the incremental MSF, that supports addition of 
edges and queries at any time, in amortized time $\Oh(\sqrt{m}\lg n)$ per 
edge insertion and MSF cost query, and uses space $\Theta(m\sqrt{m})$.  
% (Note that the update sequence length $m$ can be arbitrarily larger than~$n$, 
% so our bounds do not conflict with the lower bounds of Henzinger and Wu~\cite{HenzingerW2021}. 
% Our amortized time is better than the $\tilde{\Oh}(n)$ time bound per operation 
% from Henzinger and Wu~\cite{HenzingerW2021} for $m = o(n^2)$.)

The remainder of the paper is organized as follows. 
In Section~\ref{sec:review}, we review the strategy of Demaine, Iacono, and 
Langerman~\cite{DemaineIL2007} to transform a partially retroactive data structure 
into a fully retroactive one.
Section~\ref{sec:rebuilding} contains the description of the new proposed rebuilding step, 
its correctness proof, and its time complexity analysis. 
In Section~\ref{sec:incMSF}, we formalize the semi-retroactive incremental MSF
and, for completeness, revise how it is implemented using the proposed rebuilding approach. 
Final remarks are presented in Section~\ref{sec:final}.

\section{From partial to full retroactivity: a brief review}\label{sec:review}

Demaine, Iacono, and Langerman~\cite{DemaineIL2007} described a way to transform 
a partially retroactive data structure into a fully retroactive one.
Their result considers that the data structures use the RAM model of computation, 
and work in the pointer-machine model of Tarjan~\cite{Tarjan1979}.
They also use, in their result, the so called \emph{rollback method}, in which 
auxiliary information is stored when certain updates are performed on the data
structures, so that one can reverse these updates if needed.

For the sake of completeness, we restate their result and describe their method. 
Then, in the next section, we describe our simplified version of their result. 

\begin{theorem}[Theorem~5 in~\cite{DemaineIL2007}]
  Let $m$ denote the number of updates in the current update sequence. 
  Any partially retroactive data structure in the pointer-machine model with 
  constant indegree, supporting $T(m)$-time retroactive operations and $Q(m)$-time
  queries about the present, can be transformed into a fully retroactive data
  structure with amortized $\Oh(\sqrt{m}\,T(m))$-time retroactive operations and 
  $\Oh(\sqrt{m}\,T(m)+Q(m))$-time fully retroactive queries using $\Oh(m\,T(m))$ space.
\end{theorem}

They define $\sqrt{m}$ checkpoints $t_1,\ldots,t_{\sqrt{m}}$ and maintain $\sqrt{m}$ 
versions $D_1,\ldots,D_{\sqrt{m}}$ of the partially retroactive data structure, 
where the structure $D_i$ contains all updates that occurred before time $t_i$.
Each $t_i$ is defined so that, when $D_i$ was constructed, it contained 
the first $i\sqrt{m}$ of the $m$ updates, for $i=1,\ldots,\sqrt{m}$. 
They keep track of the entire sequence of updates.

When a retroactive operation is performed for time $t$, they perform the operation 
on all data structures $D_i$ with $t_i>t$, which costs $\Oh(\sqrt{m}\,T(m))$ time.  
When a retroactive query is made at time $t$, they find the largest $i$ such that 
$t_i \leq t$, and perform on $D_i$ all updates from the current update sequence that 
have time between $t_i$ and $t$, keeping track of auxiliary information for later rollback. 
Then they perform the query on the resulting structure, and rollback these 
updates to restore the state of the structure $D_i$ previous to the query.

They assure that, at any time, between $\sqrt{m}/2$ and $(3/2)\sqrt{m}$ updates have 
to be performed on $D_i$ to answer any query.  This implies that the time to answer a 
query is $\Oh(\sqrt{m}\,T(m)+Q(m))$.  To achieve the $\Oh(m\,T(m))$ space consumption, 
the way they assure this is by rebuilding $D_1,\ldots,D_{\sqrt{m}}$ from time to time.  
Let $m$ denote the number of updates in the update sequence when the last rebuilding took place.
In the beginning, $m=0$. 
By assumption, the partially retroactive data structure has constant indegree, so they use 
a persistent version of it, obtained according to Driscoll et al.~\cite{DriscollSST1989}.
After $\sqrt{m}/2$ retroactive operations, they update the value of $m$ and 
rebuild the persistent data structure from scratch in time $\Oh(m\,T(m))$. 
When rebuilding the persistent data structure for the current number $m$ of updates, 
they perform the sequence of $m$ updates on a fully persistent version of an 
initially empty partially retroactive data structure, and keep a pointer $D_i$ 
to the version obtained after the first $i\sqrt{m}$ updates, for $i = 1,\ldots,\sqrt{m}$. 
The retroactive updates branch off a new version of the data structure for each modified $D_i$.  
The cost for the rebuilding is therefore $\Oh(m\,T(m))$, 
which adds an amortized cost of $\Oh(\sqrt{m}\,T(m))$ per retroactive operation. 
They also argue that the space used is $\Oh(m\,T(m))$.

% Finally, the explicit justification for requiring a persistent version of the data structure, 
% as shown in the original proof for Theorem 5, is to reduce space usage. But its necessity in 
% achieving the proposed time consumption is also implicitly evident, as explained above.

\section{Rebuilding process without a persistent data structure}\label{sec:rebuilding}

In this section, we concentrate on the case in which one does not want to use a persistent data structure, 
or such a structure is not available.  We describe a rebuilding process that is as efficient, in terms of 
time, as the original one by Demaine et al.~\cite{DemaineIL2007}, but that does not use a persistent 
version of the involved data structure.  
% is simpler to implement, as it does not use a persistent version of the involved data structure. 

We refer to a data structure that gives support to retroactive queries, and to retroactive insertions 
into the update sequence, but not to removals, as a \emph{semi-retroactive} data structure.
This kind of data structure is obviously weaker than a fully retroactive one, and
is not comparable with a partially retroactive one, because the later gives support 
to queries only at the present, and to retroactive insertions and removals on the 
update sequence.  For semi-retroactive data structures, we refer to retroactive 
updates, instead of retroactive operations, as only insertion of updates are allowed. 

We will describe two variants of the rebuilding process. The first one is simpler and 
serves to derive a semi-retroactive data structure from a partially retroactive one.  
The second one serves to derive a fully retroactive data structure from a partially retroactive one. 

\subsection{Semi-retroactivity}\label{subsec:semi}

Our strategy follows the same idea of Demaine el at., but it does not 
rely on the use of a persistent version of the data structure involved. 
Also, for semi-retroactivity, we propose the use of slightly different 
checkpoints and rebuilding moments, that make it easier to implement 
and analyze the correctness of the strategy. 

Let $m$ denote the number of updates in the current update sequence. 
As we are considering semi-retroactivity, there are no removals of updates, 
and $m$ is also the number of retroactive operations that happened until now, 
that is, the total number of retroactive updates. 

We will use $D_0$ to refer to an empty data structure, 
which is the initial state of the data structure, when $m=0$. 
We will rebuild the data structures $D_0,D_1,\ldots,D_{\sqrt{m}}$ every 
time $m$ is a perfect square, that is, $m=k^2$ for a positive integer $k$.
Since $(k+1)^2-k^2 = 2k+1$, the data structures built 
when $m=k^2$ will be rebuilt after exactly $2k+1=\Theta(\sqrt{m})$ retroactive updates.

Let $S$ be the list of updates when $m=k^2$. 
Let~$S^+$ be the list of subsequent $2k+1$ updates, 
that arrived after the rebuilding that resulted in~$D_0,D_1,\ldots,D_k$,
and let $S'$ be the union of $S$ and $S^+$.  
Consider these lists sorted by the time of the updates. 

When $m=k^2$, a rebuilding occurred and $D_i$ becomes the partially retroactive 
data structure with the first $ik$ updates in $S$ for $i=0,1,\ldots,k$. 
The retroactive queries and subsequent $2k+1$ retroactive updates in $S^+$ 
are treated as in Section~\ref{sec:review}. 
When $m$ reaches $(k+1)^2$, it is time to rebuild the data structures.  
The idea is quite simple.  Let $D'_0$ and $D'_1$ be two new empty data structures 
and let $D'_{i+2}$ refer to the current $D_i$ for $i=0,1,\ldots,k-1$.  Disregard $D_k$.   
Let $t'_0 = t'_1 = 0$ and, for $i=2,\ldots,k+1$, 
let $t'_i$ be the time of the last update in $D'_i$.
For $i=1,\ldots,k+1$, apply to $D'_i$ the updates in~$S'$ 
after~$t'_i$ so that $D'_i$ stores exactly $i(k+1)$ updates.
The resulting $D'_0,D'_1,\ldots,D'_{k+1}$ are the new versions of the data structures for~$S'$. 

\medskip 

The key fact that assures that this works is the following. 

\begin{lemma}
  For $i=0,1,\ldots,k-1$, every update in $D_i$ is within 
  the first $(i+2)(k+1)$ updates for the sequence $S'$ of updates. 
\end{lemma}

\begin{proof}
  When $m=k^2$, the data structure $D_i$ contained the first $ik$ updates in~$S$. 
  Let $t_i$ be the time of the last update in $D_i$ at that moment.
  Since then, the $2k+1$ updates in $S^+$ occurred, 
  and any of them that had time $t \leq t_i$ was applied to~$D_i$. 
  Because $ik+(2k+1) < ik+i+2k+2 = (i+2)(k+1)$, 
  even if all the $2k+1$ updates in $S^+$ were applied to~$D_i$,
  all updates in $D_i$ would be among the first $(i+2)(k+1)$ updates in $S'$.
\end{proof}

Note that the statement does not hold with $i+1$ in the place of $i+2$. 
During the rebuilding, the number of updates applied to $D_i$ to get $D'_{i+2}$ 
is at most $(i+2)(k+1)-ik = 2k+2+i < 3(k+1)$, for $i=0,1,\ldots,k-1$.  
The number of updates applied to $D'_1$ is exactly $k+1$.  That is, within the 
rebuilding, $\Oh(k) = \Oh(\sqrt{m})$ updates are applied to obtain each $D'_i$.

\subsection{Full retroactivity}\label{subsec:fully}

To achieve full retroactivity, 
we must also give support to removals of updates from the update sequence.  
For this, we are not able to use the perfect squares as the moments 
of rebuilding, because the possible length of the update sequence is
not anymore related to the number of retroactive operations done so far. 
The length of the update sequence might grow and shrink over time. 
So the strategy is more similar to the original one of Demaine et al.~\cite{DemaineIL2007}. 

Let $m$ be the number of updates in the update sequence $S$ at the
moment of a rebuilding that resulted in the partially retroactive 
data structures $D_0,D_1,\ldots,D_k$, where $k=\ceil{\sqrt{m}}$.
Let $\underline{k} = \floor{\sqrt{m}}$.
Then $D_i$ contains the first $i\underline{k}$ updates in~$S$, 
for $i=1,\ldots,k-1$, and $D_k$ contains all updates in~$S$. 
We refer to the updates in $S$ as \emph{old}.

Let $\ell=1$ if $m = 0$ and $\ell=2\underline{k}-1$ if $m \geq 1$. 
After $\ell$ retroactive operations, that now might be insertions 
or removals of updates, we will rebuild the data structures.
Let $m'$ be the number of updates in the current sequence~$S'$ 
after these $\ell$ operations are performed. 
Let $k'= \ceil{\sqrt{m'}}$ and $\underline{k}' = \floor{\sqrt{m'}}$. 

\begin{claim}
  $|\underline{k}' - \underline{k}| \leq 1$.
\end{claim}
\begin{proof}
  If $m = 0$, then $m' = m + 1 = 1$, and $\underline{k}' = 1 = \underline{k}+1$.
  So suppose that $m \geq 1$, and note that $m-\ell \leq m' \leq m+\ell$.  
  Then $\sqrt{m-\ell} \leq \sqrt{m'}$.  
  But $m-\ell = m - (2\underline{k}-1) \geq m - 2\sqrt{m} + 1 = (\sqrt{m}-1)^2$, 
  because $m \geq 1$. 
  Hence $\sqrt{m-\ell} \geq \sqrt{m}-1$, 
  which implies that 
  $\underline{k}' \geq \floor{\sqrt{m-\ell}} \geq \floor{\sqrt{m}-1} = \underline{k}-1$. 
  Similarly, $\sqrt{m'} \leq \sqrt{m+\ell}$, and
  $m+\ell = m + 2\underline{k}-1 < m + 2\sqrt{m} + 1 = (\sqrt{m}+1)^2$. 
  Thus $\sqrt{m+\ell} < \sqrt{m}+1$, 
  which implies that 
  $\underline{k}' \leq \floor{\sqrt{m+\ell}} \leq \floor{\sqrt{m}+1} = \underline{k}+1$.
\end{proof}

If $\underline{k}' \geq \underline{k}$, then
$i\underline{k}+2\underline{k}-1 \leq i\underline{k}'+2\underline{k}'-1 < (i+2)\underline{k}'$.
Hence all the $i\underline{k}$ old updates that were not removed 
are within the $(i+2)\underline{k}'$ first updates in~$S'$, 
even if all the at most $2\underline{k}-1$ new updates inserted are before $t_i$. 

If $\underline{k}' = \underline{k}-1$, then $m'<m$, 
which means that at most $\underline{k}-1$ of the $2\underline{k}-1$ 
operations that occurred since the last rebuilding are insertions. 
Also, as $k' \leq \underline{k}'+1$, 
we only need to use $D_i$ to obtain $D'_{i+2}$ for $i+2 \leq k'$, 
which means that $i \leq k'-2 < \underline{k}'$.  
So, $i\underline{k}+\underline{k}-1 = i(\underline{k}'+1)+\underline{k}' 
= (i+1)\underline{k}'+i <  (i+2)\underline{k}'$. 

Hence, we can proceed essentially as in the previous subsection.
Let $D'_0$ and $D'_1$ be two new empty data structures 
and let $D'_{i+2}$ refer to the current $D_i$ for~$i=0,1,\ldots,k'-1$.  Disregard $D_k$.   
Let $t'_0 = t'_1 = 0$ and, for $i=2,\ldots,k'$, 
let~$t'_i$ be the time of the last update in $D'_i$.
For $i=1,\ldots,k'-1$, apply to $D'_i$ the updates in~$S'$ 
after~$t'_i$ so that $D'_i$ stores exactly $i\underline{k}'$ updates, 
and apply to~$D'_{k'}$ all the updates in~$S'$ after $t'_{k'}$. 
The resulting $D'_0,D'_1,\ldots,D'_{k'}$ are the new versions of the data structures for $S'$. 

Note that, within the rebuilding, 
the number of updates we perform on $D'_1$ is $\underline{k}'$, 
the number of updates we perform on $D'_{k'}$ is $m'-m \leq 2\underline{k}'-1$, 
and, for $2 \leq i \leq k'-1$, we perform at most 
$i\underline{k}' - (i-2)\underline{k}+2\underline{k}-1
 = i(\underline{k}'-\underline{k})+4\underline{k}-1 
\leq i+4\underline{k}-1 \leq k'+4\underline{k}-2 \leq 5k'+2$ updates. 
For every $i$, this number is $\Oh(k') = \Oh(\sqrt{m'})$.


The time to execute a retroactive operation remains the same, 
despite the change in the number of operations between rebuildings.
Let $m$ and $\bar{m}$ be the number of updates in the sequence at 
the last rebuilding and when an operation is done, respectively.
For a retroactive insertion or removal of an update, the amortized time 
is $\Oh(\sqrt{m}\,T(\bar{m})) = \Oh(\sqrt{\bar{m}}\,T(\bar{m}))$.  
Let $\underline{k} = \floor{\sqrt{m}}$ and $\ell = 2\underline{k}-1$. 
Because $m-\ell < \bar{m} < m+\ell$, we have that $\underline{k} = \Oh(\sqrt{\bar{m}})$. 
For a retroactive query, the number of updates applied to the 
appropriate~$D_i$ and rolled back is $\Oh(\underline{k}) = \Oh(\sqrt{\bar{m}})$, 
so the time is $\Oh(\sqrt{m}\,T(m)+Q(m))$.  


\medskip 

As for the space used by our strategy, assuming that the space used by the 
partially retroactive data structure is linear, each $D_i$ uses space $\Theta(ik)$, 
and thus the total space used by $D_0,D_1,\ldots,D_k$ is $\Theta(m\sqrt{m})$. 
The space used by Demaine et al.~\cite{DemaineIL2007} strategy, that relies 
on a persistent data structure, is $\Oh(m\,T(m))$, where $T(m)$ is the time 
for a retroactive update in the partially retroactive data structure. 

\section{Semi-retroactive incremental MSF}\label{sec:incMSF}

The approach of de Andrade Junior and Seabra~\cite{deAndradeJrS2022} for the 
semi-retroactive incremental MSF solution offers support to the following interface:
\begin{itemize}
\item \texttt{add\_edge}$(u,v,w,t)$: add to the graph $G$, at time $t$,
  an edge of cost $w$ and endpoints $u$ and $v$;
\item \texttt{get\_msf}$(t)$: return a list with the edges of an MSF of $G$ at
  time~$t$.
\end{itemize}

To implement this, one needs to keep an incremental MSF, which, in
this case, is a partial retroactive data structure. Its interface is
pretty similar to the semi-retroactive version, and the only difference
is that we drop the argument for time $t$ in both the edge addition and 
query operations. This can be implemented using link-cut trees~\cite{SleatorT1981} 
as an underlying structure for maintaining the current MSF. Specifically, 
every time a new edge $uv$ is added, we can check in the link-cut trees if 
this new edge would form a cycle with the stored forest. If not, then we 
proceed to add it to the forest.  
Otherwise, we find an edge $e$ with maximum cost on the path between $u$ 
and $v$ in the stored forest, and if the cost of the new edge $uv$ is smaller 
than the cost of~$e$, we remove $e$ and add $uv$ to the forest.
Also, when a query for MSF is performed, 
we simply return all the edges currently stored in the link-cut trees.

Following on, to implement the semi-retroactive version of the incremental MSF, 
as described in Section~\ref{sec:review}, the idea of square-root decomposition 
is used to divide the time line of length $m$ in blocks of size $\sqrt{m}$. 
Because of the restrictions imposed by de Andrade Junior and Seabra --- 
that $m$ is known beforehand and that each operation time is an integer in 
the interval $[1, m]$ --- it is possible to avoid rebuilding, and to build 
these $\sqrt{m}$ blocks right up front, as the first step in the structure 
initialization.  Each of these blocks is defined by a checkpoint $t_i$ such 
that $t_i = i \sqrt{m}$, with $i \in [1, \sqrt{m}]$. Then, each checkpoint 
$t_i$ is followed by a respective incremental MSF $D_i$, where $D_i$ has 
all the edge insertions that took place before the moment $t_i$.
An empty incremental MSF~$D_0$ is also used.

From that, the implementation of de Andrade Junior and Seabra follows 
the expected.  The operation \texttt{add\_edge}$(u,v,w,t)$ is performed 
by adding the respective edge to each $D_i$ such that $t < t_i$, for
$i \in [1,\sqrt{m}]$.  The \texttt{get\_msf}$(t)$ consists of finding 
the largest $i$ such that $t_i < t$, and then performing all the 
insertions that take place between $t_i$ and~$t$ on~$D_i$.
After that, it is possible to return the current MSF stored 
in~$D_i$ and then roll back these last performed insertions.
The empty incremental MSF~$D_0$ is used when $t$ is smaller than~$t_1$.

Now, let us take a look at the time consumption of this approach.
Recall that $n$ denotes the number of vertices of the graph, 
and therefore in the link-cut trees.  First of all, the query for the edges 
in the link-cut trees costs $\Oh(n)$, and all the other routines used from 
the link-cut trees have an amortized cost of~$\Oh(\log{n})$ per operation.  
For the \texttt{add\_edge} routine, in the worst case, we have to add one new 
edge to each~$D_i$, hence its amortized time consumption is~$\Oh(\sqrt{m}\,\log{n})$. 
Finally, the time consumption of the \texttt{get\_msf} is~$\Oh(n+\sqrt{m}\,\log{n})$,
because of the updates that need to be applied and rolled back, 
and the query for the edges in a versions of the link-cut trees.

\medskip

The development of the idea presented in this paper was driven by
the desire to overcome the limitations in de Andrade 
Junior and Seabra's solution for the semi-retroactive MSF problem.  
The main difference is that we implement the rebuilding steps, 
and hence we do not restrict the amount of operations or their time range.
The rebuilding steps are implemented following 
the approach presented in Section~\ref{subsec:semi}.
Edge insertions and queries are treated similarly to their implementation, 
but now the checkpoints change during the process, as the rebuildings happen.
% For details, please check our implementation, available at the following link: \\
% {\small \texttt{https://github.com/fcnoronha/mac0499/tree/main/implementations}}

\medskip

To emphasize the simplicity of the rebuilding step, we present below 
the idea in pseudocode, using the notation from Section~\ref{subsec:semi}.
The procedure receives an integer $k$, a sequence $D$ with the link-cut trees $D_0,\ldots,D_k$, 
the sequence $t$ where $t_i$ is the last time stamp of an edge in $D_i$ for $i=0,\ldots,k$, and
the current sequence $S$ with $(k+1)^2$ edge addition pairs $(e,s)$, stored 
for instance in a balanced binary search tree with the time stamp $s$ as key.
It returns the new block size $k+1$, the sequence $D'$ with 
the link-cut trees $D'_0,\ldots,D'_{k+1}$ and the sequence $t'$ 
where $t'_i$ is the last time stamp of an edge in~$D'_i$ for $i=0,\ldots,k+1$. 
In this pseudocode, for a pair $p=(e,s)$ in $S$, we use 
$p.\textrm{time}$ to refer to $s$. % and $p.\textrm{edge}$ to refer to $e$. 
The procedure {\sc newIncrementalMSF} returns 
a new data structure representing a spanning forest with no edges. 
It takes~$\Oh(1)$ time in our implementation. 
The procedure {\sc kth}$(S,i)$ returns the element in~$S$ with the $i$th smallest key, 
in time $\Oh(\log k)$, because $S$ has $\Oh(k^2)$ elements. 
The procedure {\sc addEdges}$(S,t_s,t_f,F)$ updates the MSF stored in~$F$ 
considering the addition of all edges in~$S$ with time stamp more than~$t_s$ 
and at most $t_f$.
It takes time $\Oh(\log k + \ell\log n)$, where $\ell$ is the number of edges added.

\medskip

\begin{algorithm}[h!]
    \caption{Rebuilding procedure}\label{rmsf-build-decomp}
    \begin{algorithmic}[1]
        \Function{rebuild}{$k,D,t,S$}
        \State $D'_0 \gets$ {\sc newIncrementalMSF}$()$
        \State $D'_1 \gets$ {\sc newIncrementalMSF}$()$
        \For{$i \gets 2$ {\bf to} $k+1$}
        \State $D'_i \gets D_{i-2}$
        \EndFor
        \State $t_{-1} \gets 0$ \hfill {\footnotesize $\rhd$ sentinel}
        \State $t'_0 \gets 0$ 
        \For{$i \gets 1$ {\bf to} $k+1$}
        \State $p \gets$ {\sc kth}$(S,i(k+1))$ \hfill {\footnotesize $\rhd$ $i(k+1)$th edge in $S$}
        \State $t'_i \gets p.\textrm{time}$ \hfill {\footnotesize $\rhd$ time stamp of the $i(k+1)$th edge in $S$}
        \State {\sc addEdges}$(S,t_{i-2},t'_i,D'_i)$
        \EndFor
        \State \Return $k+1,D',t'$
        \EndFunction
    \end{algorithmic}
\end{algorithm}

The running time is dominated by the insertion operations 
on the incremental MSFs.  As argued in Section~\ref{subsec:semi}, 
this process will execute $\Oh(m)$ such operations and, 
because each of these operations has an amortized cost of $\Oh(\log{n})$,
the total amortized cost of {\sc rebuild} is $\Oh(m \log{n})$.  This cost, distributed 
over the $\Theta(\sqrt{m})$ operations that take place between two rebuildings,
adds an $\Oh(\sqrt{m}\,\log{n})$ amortized consumption time per operation.

\section{Final remarks}\label{sec:final}

During our study of the work of de Andrade Junior and Seabra, we noticed that
they did not really implement full retroactivity, because their implementation
does not allow for removals from the update sequence.  Even though the problem
considered is incremental, a fully retroactive version of the problem should
allow for the removal of edge additions.  Note that this does not correspond
to an implementation of a retroactive dynamic MSF, because it does not keep
in the update sequence edge additions and edge removals.  The update sequence
contains only edge additions.

Our current implementation also does not give support to removals of edge additions.
The removal of an edge insertion with time stamp $t$ from the update sequence 
requires the removal of this edge from each data structure $D_i$ such that $t < t_i$. 
However, the data structure $D_i$ is for incremental MSF, and does not give support 
to edge removals. 
The algorithm of Holm, de Lichtenberg, and Thorup~\cite{HolmLT2001} maintains
dynamic graphs efficiently.  Their algorithm is also based on link-cut trees.
It would be interesting to use their ideas to achieve an implementation of
an efficient fully retroactive incremental MSF.

\bibliographystyle{elsarticle-harv}
\bibliography{retroactive.bib}

\end{document}

%%
%% End of file `procs-template.tex'.
