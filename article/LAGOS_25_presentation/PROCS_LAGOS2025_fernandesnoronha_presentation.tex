%%%%%%%%%%%%%%%%%%%%%%%%%%%%%%%%%%%%%%%%%%%%%%%%%%%%%%%%%
% LaTeX Beamer Presentation (Condensed Version)
%
% On-slide text is minimal.
% Speaker notes are included as comments (%) in each frame.
%%%%%%%%%%%%%%%%%%%%%%%%%%%%%%%%%%%%%%%%%%%%%%%%%%%%%%%%%

\documentclass[fleqn]{beamer}

\usepackage[utf8]{inputenc}
\setlength{\mathindent}{\parindent}
\usepackage{amsfonts}
\usepackage{amssymb}
\usepackage{graphicx} 
\usepackage{pifont}
\usepackage{cancel}
\usepackage{tikz}
\usetikzlibrary{calc,positioning,decorations.pathmorphing,decorations.pathreplacing}
\usetikzlibrary{backgrounds}
\usepackage{./figstyle}
\usepackage{algorithm}
\usepackage{algpseudocode}

\newcommand{\greencheck}{\textbf{\textcolor{green}{\ding{51}}}}
\newcommand{\Oh}{\mathcal{O}}

%------------------------------------------------------------
% Theme and Color Scheme
%------------------------------------------------------------
\usetheme{Boadilla}

\usecolortheme{default}

\pdfstringdefDisableCommands{%
  \def\\{}%
  \def\texttt#1{<#1>}%
  \def\vspace#1{}%
}

%------------------------------------------------------------
% Presentation Information
%------------------------------------------------------------
\title[Partial to full retroactivity]{How to go from partial to full retroactivity in detail}
\author[Felipe C. Noronha]{Cristina Gomes Fernandes, Felipe Castro de Noronha}
\institute[IME-USP]{
    IME-USP -- Brazil
}
\date{LAGOS 25 -- November 10-14, 2025}

%------------------------------------------------------------
% The Presentation Begins
%------------------------------------------------------------
\begin{document}

%-------------------------------------------------
% SLIDE 1: Title Page
%-------------------------------------------------
\begin{frame}
    \titlepage
    \note{
        - Introduce yourself and collaborators
        - State topic: going from partial to full retroactivity
    }
\end{frame}


%-------------------------------------------------
% SLIDE 2: What is a Spanning Tree?
%-------------------------------------------------
\begin{frame}
    \frametitle{What is a spanning tree?}
    
    \begin{itemize}
    \item Let $G = (V, E)$ be a connected graph \vfill
    \item \textbf{Spanning tree:} a subgraph that is a tree and contains all vertices \vfill\pause
    \item \textbf{Properties:}
        \begin{itemize}
        \item Connected (path between any two vertices)
        \item Acyclic (no cycles)
        \item Contains exactly $n-1$ edges for $n$ vertices
        \end{itemize} \vfill\pause
    \item \textbf{Minimum Spanning Tree (MST):} spanning tree with minimum total cost \vfill\pause
    \item \textbf{Minimum Spanning Forest (MSF):} generalization for disconnected graphs \vfill
    \end{itemize}
    \note{
        - Start with basic concept of spanning tree
        - Explain key properties: connected, acyclic, n-1 edges
        - Then define MST as spanning tree with minimum cost
        - Finally generalize to MSF for disconnected graphs
        - This builds up the concepts step by step
    }
\end{frame}



%-------------------------------------------------
% SLIDE 3: Incremental MSF Problem
%-------------------------------------------------
\begin{frame}
    \frametitle{Incremental MSF problem}
    
    \begin{itemize}
    \item \textbf{Problem:} Keep track of an MSF in a graph that grows over time \vfill
    \item Graph starts empty, edges are added one by one \vfill\pause
    \item \textbf{Operations:}
        \begin{itemize}
        \item \texttt{add\_edge}$(u,v,w)$: add edge with cost $w$ between vertices $u$ and $v$
        \item \texttt{get\_msf}(): return current MSF
        \end{itemize} \vfill\pause
    \item \textbf{Solution:} Frederickson (1983) using link-cut trees \vfill
    \end{itemize}
    \note{
        - Define incremental MSF problem clearly
        - Emphasize that graph starts empty and grows
        - Show the two key operations needed
        - Mention Frederickson's breakthrough solution
        - Note the cost is in terms of current number of edges
        - This sets up the foundation for retroactive version
    }
\end{frame}

%-------------------------------------------------
% SLIDE 4: Frederickson's Link-Cut Tree Solution
%-------------------------------------------------
\begin{frame}
    \frametitle{Frederickson's link-cut tree solution}
    
    \begin{itemize}
    \item \textbf{Key insight:} Use link-cut trees to maintain MSF dynamically \vfill\pause
    \item \textbf{Algorithm for adding edge $(u,v,w)$:}
        \begin{enumerate}
        \item Check if $u$ and $v$ are in same component
        \item If not: add edge to forest
        \item If yes: find max cost edge on $u$-$v$ path
        \item If $w < \text{max cost}$: replace max edge with new edge
        \end{enumerate} \vfill\pause
    \item \textbf{Link-cut tree operations:}
        \begin{itemize}
        \item \texttt{find\_max}$(u,v)$: $\Oh(\log n)$ amortized
        \item \texttt{link}$(u,v)$: $\Oh(\log n)$ amortized
        \item \texttt{cut}$(u,v)$: $\Oh(\log n)$ amortized
        \end{itemize} \vfill\pause
    \item \textbf{Total cost:} Amortized $\Oh(\log n)$ per edge addition \vfill
    \end{itemize}
    \note{
        - Explain Frederickson's key insight about using link-cut trees
        - Walk through the algorithm step by step
        - Show how cycle detection and edge replacement works
        - List the specific link-cut tree operations needed
        - Emphasize the logarithmic time complexity
        - This provides technical foundation for later slides
    }
\end{frame}

%-------------------------------------------------
% SLIDE 5: What is Retroactivity?
%-------------------------------------------------
\begin{frame}
    \frametitle{What is retroactivity?}
    
    \begin{itemize}
    \item \textbf{Problem:} Data structures usually support updates and queries \vfill
    \item Order of updates affects the state of the data structure \vfill\pause
    \item \textbf{Retroactivity:} Manipulate the sequence of updates \vfill\pause
    \item \textbf{Operations:}
        \begin{itemize}
        \item Insert update at time $t$ (possibly in the past)
        \item Remove update at time $t$
        \item Query at time $t$ (not just present)
        \end{itemize} \vfill
    \end{itemize}
    \note{
        - Start with the fundamental problem: data structures depend on update order
        - Explain the motivation: correcting mistakes, adding forgotten operations
        - Show the three key operations: insert, remove, query at any time
        - Emphasize that time stamps must be distinct
        - This sets up the distinction between partial and full retroactivity
    }
\end{frame}

%-------------------------------------------------
% SLIDE 6: Partial vs Full Retroactivity
%-------------------------------------------------
\begin{frame}
    \frametitle{Partial vs Full retroactivity}
    
    \begin{block}{Partially Retroactive}
        \begin{itemize}
        \item Queries only on \textbf{current} state
        \item Insert/remove updates at any time
        \item Example: Dynamic MSF $\rightarrow$ Partially retroactive MSF
        \end{itemize}
    \end{block} \pause
    
    \begin{block}{Fully Retroactive}
        \begin{itemize}
        \item Queries at \textbf{any} time $t$
        \item Insert/remove updates at any time
        \item Complete retroactive functionality
        \end{itemize}
    \end{block} \pause
    
    \begin{block}{Semi-Retroactive}
        \begin{itemize}
        \item Queries at \textbf{any} time $t$
        \item Insert updates at any time
        \item \textbf{No removal} of updates
        \end{itemize}
    \end{block}
    \note{
        - Clearly distinguish between partial, full, and semi-retroactivity
        - Emphasize that partial only allows queries on current state
        - Show that full allows queries at any time - much more powerful
        - Define semi-retroactive: queries at any time, insertions, but no removals
        - Give concrete example: dynamic MSF becomes partially retroactive MSF
        - This sets up the challenge in the next slide
    }
\end{frame}

%-------------------------------------------------
% SLIDE 7: The Challenge
%-------------------------------------------------
\begin{frame}
    \frametitle{The challenge}
    
    \begin{alertblock}{Challenge}
        How to transform partial $\rightarrow$ full retroactivity?
    \end{alertblock} \pause
    
    \begin{itemize}
    \item \textbf{Problem:} Need to support queries at any time $t$ \vfill
    \item \textbf{Solution approach:} Square-root decomposition \vfill\pause
    \item \textbf{Key insight:} Keep checkpoints with data structure states \vfill
    \item \textbf{Implementation:} Demaine, Iacono \& Langerman (2007) \vfill
    \end{itemize}
    \note{
        - State the main challenge clearly
        - Explain what we need to achieve: queries at any time
        - Introduce the solution approach: square-root decomposition
        - Mention the key insight about checkpoints
        - Reference the Demaine et al. work
        - This motivates the detailed solution in the next slide
    }
\end{frame}

%-------------------------------------------------
% SLIDE 8: Demaine, Iacono & Langerman's Solution
%-------------------------------------------------
\begin{frame}
    \frametitle{Demaine, Iacono \& Langerman's solution}
    
    \begin{theorem}[Theorem 05]
        Any partially retroactive data structure can be transformed into a fully retroactive one with:
        \begin{itemize}
        \item $\Oh(\sqrt{m})$ slowdown per operation
        \item $\Oh(m)$ space usage
        \item \textbf{Requirement:} Need persistent version of the data structure
        \end{itemize}
    \end{theorem} \pause
    
    \begin{itemize}
    \item \textbf{Key idea:} Square-root decomposition \vfill
    \item Keep $\sqrt{m}$ checkpoints with data structure states \vfill\pause
    \item \textbf{Query at time $t$:}
        \begin{enumerate}
        \item Find closest checkpoint before $t$
        \item Apply updates from checkpoint to $t$
        \item Answer query, then rollback
        \end{enumerate} \vfill
    \end{itemize}
    \note{
        - State Theorem 05 from Demaine, Iacono & Langerman 2007
        - Emphasize the persistent data structure requirement
        - Explain square-root decomposition concept
        - Show how queries work: find checkpoint, apply updates, rollback
        - Mention that this is the foundation for space optimization
        - Set up the problem: what if we don't have persistent version?
    }
\end{frame}

%-------------------------------------------------
% SLIDE 9: The Space Problem
%-------------------------------------------------
\begin{frame}
    \frametitle{The space problem}
    
    \begin{itemize}
    \item \textbf{Naive approach:} Keep $\sqrt{m}$ independent copies \vfill
    \item Space usage: $\Theta(m\sqrt{m})$ \vfill\pause
    \item \textbf{Demaine et al. solution:} Use persistent data structures \vfill
    \item Space usage: $\Oh(m)$ \vfill\pause
    \end{itemize}
    
    \begin{alertblock}{Problem}
        What if we don't have or don't want to use persistent data structures?
    \end{alertblock} \pause
    
    \begin{block}{Our contribution}
        Simple rebuilding strategy without persistent data structures
        \begin{itemize}
        \item Same time complexity: $\Oh(\sqrt{m})$ per operation
        \item Space usage: $\Theta(m\sqrt{m})$
        \end{itemize}
    \end{block}
    \note{
        - Explain the space issue with naive approach
        - Show how Demaine et al. solve it with persistent data structures
        - State the practical problem: persistent versions are complex
        - Present our key contribution: same performance without persistence
        - Emphasize the space trade-off we make
        - This motivates our rebuilding approach
    }
\end{frame}

%-------------------------------------------------
% SLIDE 10: Starting Point
%-------------------------------------------------
\begin{frame}
    \frametitle{Starting point}
    
    \begin{itemize}
    \item \textbf{Junior \& Seabra's solution:} Semi-retroactive incremental MSF \vfill
    \item \textbf{Operations:}
        \begin{itemize}
        \item \texttt{add\_edge}$(u,v,w,t)$: add edge at time $t$
        \item \texttt{get\_msf}$(t)$: get MSF at time $t$
        \end{itemize} \vfill\pause
    \item \textbf{Implementation:} Square-root decomposition \vfill
    \item \textbf{Checkpoints:} $t_i = i\sqrt{m}$ for $i = 1,\ldots,\sqrt{m}$ \vfill\pause
    \item \textbf{Data structures:} $D_i$ contains edges before time $t_i$ \vfill
    \item \textbf{Time:} $\Oh(\sqrt{m}\log n)$ per operation \vfill
    \end{itemize}
    \note{
        - Start with Junior & Seabra's work as our starting point
        - Explain their semi-retroactive MSF problem
        - Show their operations: add edge at time t, query at time t
        - Describe their square-root decomposition approach
        - Show how they use checkpoints and data structures
        - Mention their time complexity
        - This sets up their limitations in the next slide
    }
\end{frame}

%-------------------------------------------------
% SLIDE 11: Limitations of Junior & Seabra
%-------------------------------------------------
\begin{frame}
    \frametitle{Limitations}
    
    \begin{alertblock}{Problems with their approach}
        \begin{itemize}
        \item \textbf{Fixed $m$:} Must know sequence length beforehand
        \item \textbf{Fixed time range:} Operations must have timestamps $1$ to $m$
        \item \textbf{No rebuilding:} Cannot handle arbitrary growth
        \end{itemize}
    \end{alertblock} \pause
    
    \begin{block}{Our goal}
        Remove these limitations while maintaining efficiency
    \end{block} \pause
    
    \begin{itemize}
    \item \textbf{Key insight:} Implement rebuilding process \vfill
    \item \textbf{Challenge:} How to rebuild without persistent data structures? \vfill
    \item \textbf{Solution:} Reuse existing data structures during rebuilding \vfill
    \end{itemize}
    \note{
        - Clearly list their three main limitations
        - Emphasize that fixed m and time range are serious restrictions
        - State our goal: remove these limitations while maintaining efficiency
        - Present our key insight: implement rebuilding process
        - Explain the challenge: how to rebuild without persistent structures
        - This motivates our solution in the next slide
    }
\end{frame}

%-------------------------------------------------
% SLIDE 12: Our Solution - Rebuilding Strategy
%-------------------------------------------------
\begin{frame}
    \frametitle{Our solution - Rebuilding strategy}
    
    \begin{itemize}
    \item \textbf{Key idea:} Reuse existing data structures during rebuilding \vfill
    \item \textbf{Rebuilding moments:} When $m = k^2$ (perfect square) \vfill\pause
    \item \textbf{Strategy:}
        \begin{enumerate}
        \item Create new empty structures $D'_0, D'_1$
        \item Reuse $D_i \rightarrow D'_{i+2}$ for $i = 0,\ldots,k-1$
        \item Apply missing updates to each $D'_i$
        \end{enumerate} \vfill\pause
    \end{itemize}
    
    \begin{block}{Key Lemma}
        Every update in $D_i$ is within the first $(i+2)(k+1)$ updates in the new sequence.
    \end{block} \pause
    
    \begin{itemize}
    \item \textbf{Time per rebuilding:} $\Oh(m\log n)$ \vfill
    \item \textbf{Amortized cost:} $\Oh(\sqrt{m}\log n)$ per operation \vfill
    \end{itemize}
    \note{
        - Explain our key insight: reuse existing data structures
        - Show rebuilding moments: when m is a perfect square
        - Walk through the three-step strategy
        - Present the key lemma that makes this work
        - Analyze time complexity: O(m log n) total, O(sqrt(m) log n) amortized
        - This sets up the detailed algorithm in the next slide
    }
\end{frame}

%-------------------------------------------------
% SLIDE 13: Rebuilding Algorithm
%-------------------------------------------------
\begin{frame}
    \frametitle{Rebuilding algorithm}
    
    \begin{enumerate}
        \item $D'_0 \gets$ \textsc{newIncrementalMSF}()
        \item $D'_1 \gets$ \textsc{newIncrementalMSF}()
        \item For $i = 2$ to $k+1$: $D'_i \gets D_{i-2}$ \hfill {\footnotesize $\rhd$ reuse existing}
        \item For $i = 1$ to $k+1$:
        \begin{itemize}
        \item $p \gets$ \textsc{kth}$(S, i(k+1))$ \hfill {\footnotesize $\rhd$ $i(k+1)$th edge}
        \item $t'_i \gets p.\text{time}$
        \item \textsc{addEdges}$(S, t_{i-2}, t'_i, D'_i)$
        \end{itemize}
        \item Return $k+1, D', t'$
    \end{enumerate} \pause
    
    \begin{figure}
        \centering
        \begin{tikzpicture}[scale=0.8]
            % Original checkpoints
            \node[draw, rectangle, minimum width=1.5cm, minimum height=0.8cm] (D0) at (2, 2) {$D_0$};
            \node[draw, rectangle, minimum width=1.5cm, minimum height=0.8cm] (D1) at (4, 2) {$D_1$};
            \node[draw, rectangle, minimum width=1.5cm, minimum height=0.8cm] (D2) at (6, 2) {$D_2$};
            \node[draw, rectangle, minimum width=1.5cm, minimum height=0.8cm] (D3) at (8, 2) {$D_3$};
            \node[draw, rectangle, minimum width=1.5cm, minimum height=0.8cm] (D4) at (10, 2) {$D_4$};
            \node[draw, rectangle, minimum width=1.5cm, minimum height=0.8cm] (D5) at (12, 2) {$D_5$};
            
            % New checkpoints
            \node[draw, rectangle, minimum width=1.5cm, minimum height=0.8cm, fill=blue!20] (D0new) at (0, 0) {$D'_0$};
            \node[draw, rectangle, minimum width=1.5cm, minimum height=0.8cm, fill=blue!20] (D1new) at (2, 0) {$D'_1$};
            \node[draw, rectangle, minimum width=1.5cm, minimum height=0.8cm, fill=green!20] (D2new) at (4, 0) {$D'_2$};
            \node[draw, rectangle, minimum width=1.5cm, minimum height=0.8cm, fill=green!20] (D3new) at (6, 0) {$D'_3$};
            \node[draw, rectangle, minimum width=1.5cm, minimum height=0.8cm, fill=green!20] (D4new) at (8, 0) {$D'_4$};
            \node[draw, rectangle, minimum width=1.5cm, minimum height=0.8cm, fill=green!20] (D5new) at (10, 0) {$D'_5$};
            \node[draw, rectangle, minimum width=1.5cm, minimum height=0.8cm, fill=green!20] (D6new) at (12, 0) {$D'_6$};
            
            % Arrows showing reuse
            \draw[->, thick, red] (D0.south) -- (D2new.north);
            \draw[->, thick, red] (D1.south) -- (D3new.north);
            \draw[->, thick, red] (D2.south) -- (D4new.north);
            \draw[->, thick, red] (D3.south) -- (D5new.north);
            \draw[->, thick, red] (D4.south) -- (D6new.north);
            
            % Labels
            \node at (0, 3) {\small Original};
            \node at (0, -1) {\small New};
            \node at (6, -1.5) {\small $D_i \rightarrow D'_{i+2}$};
        \end{tikzpicture}
        \caption{Rebuilding: Reuse existing data structures with offset}
    \end{figure}
    
    \begin{itemize}
    \item \textbf{Key insight:} Reuse $D_i \rightarrow D'_{i+2}$ \vfill
    \item \textbf{Time:} $\Oh(m\log n)$ total, $\Oh(\sqrt{m}\log n)$ amortized \vfill
    \item \textbf{Space:} $\Theta(m\sqrt{m})$ \vfill
    \end{itemize}
    \note{
        - Show the step-by-step rebuilding algorithm
        - Explain how we create new empty structures D'_0, D'_1
        - Show how we reuse existing structures with offset
        - Walk through the process of applying missing updates
        - Explain the key insight: D_i becomes D'_{i+2}
        - Analyze time complexity: O(m log n) total, O(sqrt(m) log n) amortized
        - This leads to our results in the next slide
    }
\end{frame}

%-------------------------------------------------
% SLIDE 14: Results
%-------------------------------------------------
\begin{frame}
    \frametitle{Results}
    
    \begin{block}{Our contribution}
        \begin{itemize}
        \item \textbf{General transformation:} Partial $\rightarrow$ Full retroactivity \vfill
        \item \textbf{No persistent data structures needed} \vfill
        \item \textbf{Same time complexity:} $\Oh(\sqrt{m})$ per operation \vfill
        \item \textbf{Space trade-off:} $\Theta(m\sqrt{m})$ vs $\Oh(m)$ \vfill
        \end{itemize}
    \end{block} \pause
    
    \begin{block}{Semi-retroactive MSF implementation}
        \begin{itemize}
        \item \textbf{Operations:} \texttt{add\_edge}$(u,v,w,t)$, \texttt{get\_msf}$(t)$ \vfill
        \item \textbf{Time:} $\Oh(\sqrt{m}\log n)$ per operation \vfill
        \item \textbf{Space:} $\Theta(m\sqrt{m})$ \vfill
        \item \textbf{No fixed $m$ or time range restrictions} \vfill
        \end{itemize}
    \end{block}
    \note{
        - Summarize our main theoretical contribution
        - Emphasize that we don't need persistent data structures
        - Show we achieve the same time complexity as Demaine et al.
        - Present our MSF implementation results
        - Highlight that we removed the fixed m and time range restrictions
        - This demonstrates the practical value of our approach
    }
\end{frame}

%-------------------------------------------------
% SLIDE 15: Extending for Full Retroactivity
%-------------------------------------------------
\begin{frame}
    \frametitle{Extending for full retroactivity}
    
    \begin{itemize}
    \item \textbf{General applicability:} Works for any partially retroactive data structure \vfill\pause
    \item \textbf{Supporting removals:} To achieve full retroactivity \vfill
        \begin{itemize}
        \item Adapt rebuilding trigger: when $|\lfloor\sqrt{m'}\rfloor - \lfloor\sqrt{m}\rfloor| \leq 1$
        \item Handle both insertions and removals in update sequence
        \item Rebuilding frequency: every $2\lfloor\sqrt{m}\rfloor-1$ operations
        \end{itemize} \vfill\pause
    \item \textbf{Requirements:}
        \begin{itemize}
        \item Partially retroactive data structure
        \item Rollback capability
        \item No persistent version needed
        \end{itemize} \vfill
    \end{itemize}
    \note{
        - Emphasize the general applicability of our approach
        - Explain how to extend for full retroactivity with removals
        - Show the adapted rebuilding trigger condition
        - Explain how to handle both insertions and removals
        - List the requirements: partially retroactive, rollback capability
        - This shows how our approach can be extended for full functionality
    }
\end{frame}

%-------------------------------------------------
% SLIDE: Thank You
%-------------------------------------------------
\begin{frame}
    \begin{center}
        \Huge{Thank you! \\ \vfill Questions?}
    \end{center}
    \note{
        - Invite questions
        - Go back to previous slide
    }
\end{frame}

\end{document}