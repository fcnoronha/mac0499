%!TeX root=../monografia.tex

%% ------------------------------------------------------------------------- %%
\chapter{Floresta Geradora Minima incremental}
\label{cap:incremental-msf}

Neste capítulo, falaremos do problema da floresta geradora mínima incremental --- \emph{incremental minimum spanning forest}, em inglês. A solução deste problema é utilizada por \citet{10.1093/comjnl/bxaa135} para implementar a versão retroativa da floresta geradora mínima, que estudaremos no próximo capítulo.

%% ------------------------------------------------------------------------- %%
\section{Ideia}
\label{sec:imsf-ideia}

Primeiramente, a árvore geradora minima de uma grafo é um conjunto de arestas que conecta todos os vértices do grafo e tem peso mínimo. Em geral, caso o grafo não seja conexo, a floresta geradora mínima é o conjunto de árvores geradoras mínimas de cada uma das componentes do grafo.

Em linhas gerais, para resolver este problema, queremos uma estrutura que consiga manter um grafo não direcionado, com pesos nas arestas e que está sempre sofrendo a adição de novas arestas. Essa estrutura também deve ser capaz de calcular, de maneira eficiente, a floresta geradora mínima deste grafo. Desta forma, estamos interessados na seguinte interface.

\begin{itemize}
    \item \texttt{add\_edge(u, v, w)}: adiciona no grafo a aresta com pontas em $u$ e $v$ com peso $w$, possivelmente alterando a floresta geradora mínima.
    \item \texttt{get\_msf()}: retorna uma lista com todas as arestas que compõem a floresta geradora mínima no momento atual.
    \item \texttt{get\_msf\_cost()}: retorna o custo da floresta geradora mínima no momento atual.
\end{itemize}

Com isso, é possível construir um grafo de maneira incremental --- isto é, adicionando aresta por aresta --- tendo sempre em mãos a sua respectiva floresta geradora mínima. Tudo isso com um custo $\Oh (\log n)$ para a adição de novas arestas, um custo linearmente proporcional ao tamanho da floresta geradora para a consulta das arestas que a compõem e um custo constante para a consulta do custo total da floresta.

%% ------------------------------------------------------------------------- %%
\section{Estrutura interna}
\label{sec:imsf-est-int}

Assim como no union-find retroativo, vamos utilizar a link-cut tree como a estrutura interna da solução deste problema. Para isso, queremos que a link-cut tree seja utilizada para manter a floresta geradora mínima do grafo corrente. De modo que, ao adicionarmos uma nova aresta, com peso $w$ e pontas em $u$ e $v$, ao grafo, podemos usar as rotinas \texttt{is\_connected(u,v)} e \texttt{maximum\_edge(u,v)} para decidir se incluímos ou não a aresta à floresta geradora mínima.

Um detalhe importante é que para essa implementação, necessitamos de uma maneira de consultar qual a aresta com maior custo no caminho entre dois vértices na link-cut tree, não apenas o seu respectivo valor. Para isso, adicionamos um novo parâmetro opcional \texttt{id} na rotina \texttt{link}, além do novo método \texttt{maximum\_edge\_id}, que retorna o \texttt{id} da maior aresta no caminho entre dois vértices. Este \texttt{id} sera definido por nossa estrutura, e a partir dele, utilizando um mapa \texttt{edges\_by\_id}, conseguimos recuperar em quais vértices tal aresta incide.

Finalmente, mantemos uma lista \texttt{current\_msf} de \texttt{id}'s das arestas que compõem a floresta geradora minima, assim como um inteiro \texttt{current\_msf\_cost}, que armazena o respectivo custo. Estes atributos nos permitem responder de maneira eficiente as consultas de nossa estrutura, como mostraremos a seguir.

%% ------------------------------------------------------------------------- %%
\section{Consultas Get MSF e Get MST Cost}
\label{sec:imsf-get-msf}

Primeiramente, para realizarmos a consulta acerca da composição da floresta geradora mínima, simplesmente percorremos a lista dos \texttt{id}'s das arestas que compõem a floresta e criamos uma nova lista com as arestas em si, utilizando o mapeamento fornecido pelo \texttt{edges\_by\_id[id]}.

\begin{algorithm}[h!]
    \caption{Consulta Get MSF}\label{imsf-get-msf}
    \begin{algorithmic}
        \Function{get\_msf}{}
        \State \emph{msf $\gets$ []}
        \ForEach{\emph{id in current\_msf}}
        \State \emph{msf.append(edges\_by\_id[id])}
        \EndFor
        \State \Return \emph{msf}
        \EndFunction
    \end{algorithmic}
\end{algorithm}

Já a consulta sobre o custo da floresta geradora mínima pode ser facilmente respondida retornando o inteiro \texttt{current\_msf\_cost} mantido pela rotina \texttt{add\_edge}, explicada na seção a seguir.

\begin{algorithm}[h!]
    \caption{Consulta Get MSF Cost}\label{imsf-get-msf-cost}
    \begin{algorithmic}
        \Function{get\_msf\_cost}{}
        \State \Return \emph{current\_msf\_cost}
        \EndFunction
    \end{algorithmic}
\end{algorithm}

Com isso, a primeira consulta tem um custo proporcional a $\Oh(m)$, onde $m$ é o número de arestas inseridas no grafo, pois no pior caso o grafo pode ser a própria floresta geradora mínima e a segunda consulta tem um custo $\Oh(1)$.

%% ------------------------------------------------------------------------- %%
\section{Rotina Add Edge}
\label{sec:imsf-add-edge}

Como a parte mais importante da nossa estrutura, a rotina \texttt{add\_edge(u, v, w)} é responsável por adicionar uma nova aresta ao grafo, possivelmente modificando a sua respectiva floresta geradora mínima.







% falar das duas rotinas extras necessarias para a versao retroativa
