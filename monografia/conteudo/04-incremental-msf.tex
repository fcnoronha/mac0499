%!TeX root=../monografia.tex

%% ------------------------------------------------------------------------- %%
\chapter{Floresta Geradora Mínima Incremental}
\label{cap:incremental-msf}

Neste capítulo, falaremos do problema da floresta geradora mínima incremental --- \emph{incremental minimum spanning forest}, em inglês. A solução deste problema é utilizada por \citet{10.1093/comjnl/bxaa135} para implementar uma versão semi-retroativa da floresta geradora mínima, que estudaremos no próximo capítulo.

%% ------------------------------------------------------------------------- %%
\section{Ideia}
\label{sec:imsf-ideia}

O \emph{problema da árvore geradora mínima} consiste no seguinte: dado um grafo conexo $G$, com um peso associado a cada aresta, determinar uma árvore geradora de peso mínimo. Note que, este problema admite tanto pesos positivos quanto negativos nas arestas. Caso o grafo de entrada seja desconexo, buscamos uma floresta maximal de peso mínimo, que consiste em uma árvore geradora mínima de cada componente conexa do grafo.

Algoritmos como o de \citet{6773228} ou o de \citet{kruskal1956shortest} são famosos por resolver este problema de maneira eficiente, ambos com complexidade de $\Oh(|E| \log |V|)$, onde $V$ é o conjunto de vértices e $E$ o conjunto de arestas do grafo.

Já na versão incremental do problema, inicialmente sabemos apenas o número de vértices do grafo, que começa sem nenhuma aresta. Em seguida, uma a uma, as arestas são inseridas, cada uma com um dado peso. Devemos, sempre que for requisitado, fornecer eficientemente uma floresta maximal de peso mínimo do grafo corrente.

Uma solução ingênua para esta versão seria acionar o algoritmo de Prim ou o de Kruskal a cada consulta. Porém, essa alternativa seria muita cara, pois ela não considera que entre uma consulta e outra o grafo pode ter mudado muito pouco. A ideia então é utilizar a versão apresentada por \citet{frederickson1985data} para mantermos informações sobre o grafo, de modo a conseguirmos responder às consultas de maneira eficiente.

Desta forma, a estrutura de dados que vamos apresentar dá suporta à seguinte interface:

\begin{itemize}
    \item \texttt{add\_edge(u, v, w)}: adiciona no grafo a aresta com pontas em $u$ e $v$, e peso $w$.
    \item \texttt{get\_msf()}: retorna a lista com todas as arestas que compõem uma floresta maximal de peso mínimo do grafo corrente.
    \item \texttt{get\_msf\_weight()}: retorna o custo de uma floresta maximal de peso mínimo do grafo corrente.
\end{itemize}

A partir destes métodos, é possível construir um grafo de maneira incremental, isto é, adicionando aresta por aresta, com o advento de termos sempre em mãos uma respectiva floresta maximal de peso mínimo. Em particular, a rotina \texttt{add\_edge} consumirá tempo $\Oh (\log n)$ amortizado por operação --- onde $n$ é o número de vértices do grafo, a rotina \texttt{get\_msf} consumirá tempo $\Oh (\log n)$ e \texttt{get\_msf\_weight} será executada em tempo constante.

%% ------------------------------------------------------------------------- %%
\section{Estrutura interna}
\label{sec:imsf-est-int}

Assim como no union-find retroativo, vamos utilizar as \emph{link-cut trees} como estrutura interna da solução deste problema. Para isso, queremos que as \emph{link-cut trees} sejam utilizadas para manter uma floresta maximal de peso mínimo do grafo corrente, de modo que, ao adicionarmos uma nova aresta, com peso $w$ e pontas em $u$ e $v$, ao grafo, possamos usar as rotinas \texttt{is\_connected(u,v)} e \texttt{maximum\_edge(u,v)} para decidir se incluímos ou não a aresta à floresta maximal de peso mínimo.

Um detalhe importante é que, para essa implementação, necessitamos de uma maneira de consultar qual a aresta com maior peso no caminho entre dois vértices numa árvore, não apenas o seu peso. Para isso, modificamos a implementação das \emph{link-cut trees} para incluir um novo parâmetro opcional \texttt{id} na rotina \texttt{link}, além de um novo  método \texttt{maximum\_edge\_id}, que retorna o \texttt{id} de uma aresta de peso máximo no caminho entre dois vértices. Este \texttt{id} será definido por nossa estrutura, e a partir dele, utilizando uma tabela de \emph{hash} \texttt{edges\_by\_id}, conseguimos recuperar em quais vértices tal aresta incide.

Finalmente, mantemos uma lista \texttt{current\_msf} de \texttt{id}'s das arestas que compõem uma floresta maximal de peso mínimo, assim como um inteiro \texttt{current\_msf\_weight}, que armazena o seu custo. Estes atributos nos permitem responder de maneira eficiente às consultas, como mostraremos a seguir.

%% ------------------------------------------------------------------------- %%
\section{Consultas Get MSF e Get MSF Weight}
\label{sec:imsf-get-msf}

Primeiramente, para realizarmos a consulta acerca da composição de uma floresta maximal de peso mínimo, simplesmente percorremos a lista dos \texttt{id}'s das arestas que compõem a floresta armazenadas nas \emph{link-cut trees} e criamos uma nova lista com as arestas em si, utilizando o mapeamento fornecido pela tabela \texttt{edges\_by\_id}.

\begin{algorithm}[h!]
    \caption{Consulta Get MSF}\label{imsf-get-msf}
    \begin{algorithmic}[1]
        \Function{get\_msf}{}
        \State {msf $\gets$ []}
        \ForEach{{id in current\_msf}}
        \State {msf.append(edges\_by\_id[id])}
        \EndFor
        \State \Return {msf}
        \EndFunction
    \end{algorithmic}
\end{algorithm}

Já a consulta sobre o custo de uma floresta maximal de peso mínimo pode ser facilmente respondida retornando o inteiro \texttt{current\_msf\_weight}, mantido pela rotina \texttt{add\_edge}.

\begin{algorithm}[h!]
    \caption{Consulta Get MSF Weight}\label{imsf-get-msf-weight}
    \begin{algorithmic}[1]
        \Function{get\_msf\_weight}{}
        \State \Return {current\_msf\_weight}
        \EndFunction
    \end{algorithmic}
\end{algorithm}

Com isso, a consulta \texttt{get\_msf} tem um custo proporcional a $\Oh(min(m, n))$, onde $n$ é o número de vértices e $m$ é o número de arestas no grafo. Já a consulta \texttt{get\_msf\_weight} tem um custo $\Oh(1)$.

%% ------------------------------------------------------------------------- %%
\section{Rotina Add Edge}
\label{sec:imsf-add-edge}

Como a parte mais importante da estrutura, a rotina \texttt{add\_edge(u, v, w)} é responsável por adicionar uma nova aresta ao grafo, com extremos em $u$ e $v$ e peso $w$, possivelmente tendo que atualizar a floresta maximal de peso mínimo. Este processo pode ser divido em dois casos.

O primeiro deles ocorre quando $u$ e $v$ pertencem a componentes distintas do grafo. Neste caso, simplesmente adicionamos a aresta $uv$ à floresta maximal de peso mínimo, ou seja, à floresta representada pelas \emph{link-cut trees}.

O segundo caso ocorre quando $u$ e $v$ fazem parte da mesma componente do grafo. Neste caso, devemos decidir se essa aresta $uv$ deve ou não substituir alguma aresta na árvore geradora mínima dessa componente. Note que, se adicionarmos essa nova aresta na árvore, criaremos um ciclo, que consiste em todas as arestas no caminho de $u$ até $v$ na árvore, mais a nova aresta $uv$. Ademais, a adição da aresta $uv$ somente faz sentido caso ela diminua o custo total da árvore, em outras palavras, caso ela não seja a maior aresta deste ciclo. Dessa forma podemos simplesmente excluir a aresta com maior peso do ciclo, que pode ser $uv$ ou alguma outra, preservando a estrutura de árvore e possivelmente contribuindo para uma diminuição de seu custo total. Note que, este ciclo nunca vai ser criado na estrutura, apenas consultamos o valor da maior aresta entre $u$ e $v$, caso eles estejam conectados, e fazemos a devida comparação entre este valor e o da aresta $uv$.

\begin{algorithm}[h!]
    \caption{Rotina Add Edge}\label{imsf-add-edge}
    \begin{algorithmic}[1]
        \Function{add\_edge}{{u, v, w}}
        \State {edge\_id $\gets$ create\_unique\_edge\_id()}
        \State {edges\_by\_id[edge\_id] $\gets$ new edge(u, v, w, edge\_id)}
        \IfNot{{linkCutTree.is\_connected(u, v)}}
        \State {linkCutTree.link(u, v, w, edge\_id)}
        \State {current\_msf.append(edge\_id)}
        \State {current\_msf\_weight $\gets$ current\_msf\_weight + w}
        \ElsIf{{linkCutTree.maximum\_edge(u, v) $>$ w}}
        \State {maximum\_edge\_id $\gets$ linkCutTree.maximum\_edge\_id(u, v)}
        \State {maximum\_edge $\gets$ edges\_by\_id[maximum\_edge\_id]}
        \State {linkCutTree.cut(maximum\_edge.u, maximum\_edge.v)}
        \State {current\_msf.erase(maximum\_edge.id)}
        \State {current\_msf\_weight $\gets$ current\_msf\_weight - maximum\_edge.w}
        \State {linkCutTree.link(u, v, w, edge\_id)}
        \State {current\_msf.append(edge\_id)}
        \State {current\_msf\_weight $\gets$ current\_msf\_weight + w}
        \EndIf
        \EndFunction
    \end{algorithmic}
\end{algorithm}

Com isso, como esta rotina usa apenas os métodos fornecidos pelas \emph{link-cut trees}, podemos concluir que ela consome tempo amortizado proporcional a $\Oh(\log n)$, onde $n$ é o número de vértices do grafo representado.

%% ------------------------------------------------------------------------- %%
\section{Versão dinâmica}
\label{sec:versao-dinamica}

Além da versão incremental do problema, que apresentamos neste capítulo, existe uma versão dinâmica, onde é permitida também a remoção de arestas do grafo. O trabalho de \citet{hanauer2021recent} cita algumas soluções para essa versão do problema. Em particular, \citet{10.1145/502090.502095} propõem uma solução com custo amortizado $\Oh(\log^4 n)$, onde $n$ é o número de vértices do grafo. Porém, essa solução é bastante sofisticada e seu estudo fugiria do escopo deste trabalho.