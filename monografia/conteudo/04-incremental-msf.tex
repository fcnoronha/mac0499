%!TeX root=../monografia.tex

%% ------------------------------------------------------------------------- %%
\chapter{Floresta Geradora Minima incremental}
\label{cap:incremental-msf}

Neste capítulo, falaremos do problema da floresta geradora mínima incremental --- \emph{incremental minimum spanning forest}, em inglês. A solução deste problema é utilizada por \citet{10.1093/comjnl/bxaa135} para implementar a versão retroativa da floresta geradora mínima, que estudaremos no próximo capítulo.

%% ------------------------------------------------------------------------- %%
\section{Ideia}
\label{sec:imsf-ideia}

Primeiramente, a árvore geradora minima de uma grafo é um conjunto de arestas que conecta todos os vértices do grafo e tem peso mínimo. Em geral, caso o grafo não seja conexo, a floresta geradora mínima é o conjunto de árvores geradoras mínimas de cada uma das componentes do grafo.

Em linhas gerais, para resolver este problema, queremos uma estrutura que consiga manter um grafo não direcionado, com pesos nas arestas e que está sempre sofrendo a adição de novas arestas. Essa estrutura também deve ser capaz de calcular, de maneira eficiente, a floresta geradora mínima deste grafo. Desta forma, estamos interessados na seguinte interface.

\begin{itemize}
    \item \texttt{add\_edge(u, v, w)}: adiciona no grafo a aresta com pontas em $u$ e $v$ com peso $w$, possivelmente alterando a floresta geradora mínima.
    \item \texttt{get\_msf()}: retorna uma lista com todas as arestas que compõem a floresta geradora mínima no momento atual.
    \item \texttt{get\_msf\_cost()}: retorna o custo da floresta geradora mínima no momento atual.
\end{itemize}

Com isso, é possível construir um grafo de maneira incremental --- isto é, adicionando aresta por aresta --- tendo sempre em mãos a sua respectiva floresta geradora mínima. Tudo isso com um custo $\Oh (\log n)$ para a adição de novas arestas, um custo linearmente proporcional ao tamanho da floresta geradora para a consulta das arestas que a compõem e um custo constante para a consulta do custo total da floresta.

%% ------------------------------------------------------------------------- %%
\section{Estrutura interna}
\label{sec:imsf-est-int}



% falar da adaptacao feita na LCT para devolver o id de um no com o maior pesos

% falar das duas rotinas extras necessarias para a versao retroativa
