%!TeX root=../monografia.tex

%% ------------------------------------------------------------------------- %%
\chapter{Union-Find}
\label{cap:union-find}

Neste capítulo falaremos do union-find retroativo, introduzida por \citet{10.1145/1240233.1240236}, ela será primeira estrutura retroativa que vamos implementar usando a link-cut tree.

%% ------------------------------------------------------------------------- %%
\section{Ideia}
\label{sec:uf-ideia}

O union-find é uma estrutura de dados utilizada para manter uma coleção de conjuntos disjuntos, isto é, que não se intersectam. Para isso, ela fornece duas operações principais, são elas:

\begin{itemize}
    \item \texttt{same\_set(a, b)}: retorna \emph{verdadeiro} caso $a$ e $b$ estejam no mesmo conjunto, \emph{falso} caso contrario.
    \item \texttt{union(a, b)}: se $a$ e $b$ estão em conjuntos distintos, realiza a união destes conjuntos.
\end{itemize}

A primeira versão do union-find foi apresentada por \citet{10.1145/364099.364331}. Posteriormente, \citet{10.1145/62.2160}, utilizam a técnica de compressão de caminhos para mostrar uma implementação com complexidade de $\Oh( \alpha (n) )$, onde $n$ é o número total de elementos nos conjuntos que estamos representando e $\alpha()$ é o inverso da função de Ackermann.

Na versão retroativa, estamos interessados em realizar as operações em uma linha de tempo, isto é, conseguirmos adicionar e remover operações \texttt{union} em certos instantes de tempo, assim como checar se dois elementos pertenciam a dado conjunto em um certo instante $t$.

Para isso, vamos trocar a operação \texttt{union(a, b)} por duas novas rotinas, \texttt{create\_union(a, b, t)} e \texttt{delete\_union(t)}. A primeira delas é responsável por criar uma união dos conjuntos que contém $a$ e $b$ no instante de tempo $t$, enquanto a segunda desfaz a união realizada em $t$. Além disso, colocamos um terceiro parâmetro $t$ na operação \texttt{same\_set}, com isso, conseguimos especificar o instante que estamos interessados em consultar.

Note que, em nenhum momento podemos fazer uma operação que seria invalida em qualquer instante de tempo. Em outras palavras, não podemos remover uma união que não aconteceu, assim como não podemos criar uma união em dois elementos que já pertencem ao mesmo conjunto.

%% ------------------------------------------------------------------------- %%
\section{Consultas Same Set}
\label{sec:uf-same-set}


%% ------------------------------------------------------------------------- %%
\section{Rotinas Create Union e Delete Union}
\label{sec:uf-union}