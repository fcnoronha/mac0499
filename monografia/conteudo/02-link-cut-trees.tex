%!TeX root=../tese.tex
%("dica" para o editor de texto: este arquivo é parte de um documento maior)
% para saber mais: https://tex.stackexchange.com/q/78101/183146

%% ------------------------------------------------------------------------- %%
\chapter{Link-Cut Trees}
\label{cap:link-cut-trees}

Neste capítulo, apresentaremos a estrutura de dados chamada de Link-Cut Tree. Introduzida por ~\citet{10.1145/800076.802464}, essas arvores nos permitem realizar três operações principais, são elas:

\begin{itemize}
    \item $make\_root(u)$: enraíza no vértice $u$ a arvore que o contem.
    \item $link(u, v)$: dado que $u$ e $v$ estão em arvores separadas, transforma $v$ em raiz e o liga como filho de $u$.
    \item $cut(u, v)$: retira da arvore a aresta com pontas em $u$ e $v$, criando duas novas arvores.
\end{itemize}

Além disso, a Link-Cut Tree possui a capacidade de realizar operações agregadas nos vertices, isto é, consultas acerca de propriedades de uma sub-arvore ou de um caminho entre dois vertices. Em particular, para as estruturas que estudaremos nos próximos capítulos, vamos utilizar a função $maximum\_edge(u, v)$, que nos informa o valor máximo de uma aresta entre os vertices $u$ e $v$.

Para que essas operações sejam realizadas de maneira rápida, usamos a ideia de \textit{preferred paths}. No artigo original, os autores utilizam uma arvore binaria enviesada para cuidar destes caminhos, porém, 4 anos depois, os autores apresentam a Splay Tree, que possibilita realizarmos as operações necessárias para manter os \textit{preferred paths} em tempo $O(logn)$ amortizado.

%% ------------------------------------------------------------------------- %%
\section{Splay Trees}
\label{sec:splay-trees}

Também introduzida por ~\citet{10.1145/3828.3835}, a Splay Tree é uma arvore binaria de busca auto-ajustável, capaz de realizar as operações de inserção, deleção e busca em tempo $O(lg n)$ amortizado.
