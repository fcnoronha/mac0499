%!TeX root=../monografia.tex

%% ------------------------------------------------------------------------- %%
\chapter{Floresta Geradora Mínima retroativa}
\label{cap:retroactive-msf}

Neste capitulo, estudaremos a solução apresentada por \citet{10.1093/comjnl/bxaa135} para o problema da floresta geradora mínima retroativa --- \emph{retroactive minimum spanning forest}, em inglês. Esta versão, utiliza a técnica de \emph{square-root decomposition} junto com a estrutura do capítulo anterior para solucionar o problema, oferecendo uma alternativa mais simples, porém mais limitada, a outras implementações na literatura, como a de \citet{10.1145/502090.502095}.

%% ------------------------------------------------------------------------- %%
\section{Square-root decomposition}
\label{sec:sqrt-decomp}

Inicialmente, vamos conhecer a técnica de \emph{square-root decomposition}, utilizada para transformar operações que consomem tempo proporcional a $\Oh(\log n)$ --- onde $n$ é o número de elementos no problema em questão --- em operações que gastam $\Oh(\sqrt{n})$. Para nossa explicação, vamos utilizar o seguinte problema como exemplo: dado uma lista de inteiros $ [ a_1, a_2, a_3, \dots, a_n ] $, queremos conseguir efetuar as duas operações a seguir:

\begin{itemize}
    \item \texttt{find\_sum(l, r)}: encontra a soma de todos os valores no intervalo $[l,r]$;
    \item \texttt{update\_value(i, x)}: atualiza o valor da posição $i$ para $x$.
\end{itemize}

Este problema possui duas soluções \emph{ingênuas}, cada uma favorecendo uma das operações. A primeira, e mais simples, consiste em utilizar um \emph{loop} para responder consultas \texttt{find\_sum}, o que acaba custando $\Oh(n)$, e apenas atualizando a respectiva posição para a operação \texttt{update\_value}, o que consome tempo $\Oh(1)$.

Já a segunda solução se resume a utilizarmos um vetor de soma de prefixos --- isto é, um vetor tal que \texttt{prefix\_sum[i]} equivale a $\Sigma_{j=1}^{i} a_j$ --- para respondermos as consultas \texttt{find\_sum} em tempo constante, porém, acarretando na reconstrução de \texttt{prefix\_sum} em toda chamada de \texttt{update\_value}, o que consome $\Oh(n)$.

Agora, podemos partir para explicar como funciona a solução que utiliza a \emph{square-root decomposition}. O cerne desta técnica consiste em duas etapas. Primeiramente, dividimos a estrutura de interesse --- neste caso, uma lista --- em $d$ blocos de tamanho $b$. Sem perda de generalidade, assumimos que $n$, o tamanho da lista, é um múltiplo de $b$, com $d = \frac{n}{b}$. Em seguida, para cada um dos blocos, pré-calculamos o resultado do problema de interesse de seus elementos. No problema utilizado como exemplo, isso se traduz em pré-calcular a soma de todos os elementos dentro de um bloco.

Com isso, podemos explicar como adaptamos as operações para funcionarem utilizando esta divisão em blocos. Note que, apesar de estarmos focados em resolver o problema de soma em um intervalo, a \emph{receita} por trás dessa adaptação pode ser facilmente utilizada em outros contextos, como veremos na próxima seção.

Para respondermos consultas do tipo \texttt{find\_sum(l, r)}, utilizaremos o pré-calculo realizado nos blocos para eliminar a necessidade de percorrer todos os elementos no intervalo entre $l$ e $r$. Primeiramente, iteramos sob todos os blocos completamente contidos no intervalo e acumulamos a respectiva soma em uma variável $y$. Com isso em mãos, podemos nos concentrar para calcular a soma $x$ e $z$ das \emph{pontas} do intervalo, isto é, os pedaços que faziam parte de um bloco não totalmente contido no intervalo. Esta tarefa está representada na figura \ref{fig:sqrt-decomp-blocks} e podemos perceber que a reposta para a consulta é simplesmente a soma $x + y + z$.

Já a rotina \texttt{update\_value(i, x)} é um pouco mais simples. Ao atualizamos um valor na posição $i$, temos simplesmente que atualizar o valor pré-calculado do bloco que o contém, e isso é o suficiente.

\begin{figure}
    \centering
    \begin{equation*}
        \underbracket{a_1, a_2, \overbrace{a_3, a_4}^x}_{block_1} \quad
        \overbrace{
            \underbracket{a_5, a_6, a_7, a_8}_{block_2} \quad
            \dots \quad
            \underbracket{a_{n-7}, a_{n-6}, a_{n-5}, a_{n-4}}_{block_{d-1}}
        }^y \quad
        \underbracket{\overbrace{a_{n-3}, a_{n-2}, a_{n-1}}^z, a_n}_{block_d}
    \end{equation*}
    \caption{Divisão de uma lista de tamanho $n$ em $d$ blocos de tamanho $b$, mostrando que a soma de $x$, $y$ e $z$ reponde a consulta feita por \texttt{find\_sum(3, n-1)}.}
    \label{fig:sqrt-decomp-blocks}
\end{figure}

Note que, a segunda operação tem um custo constante, dado que apenas atualizamos um único valor, porém a segunda operação requer uma analise mais cuidadosa. Para encontrar os valores das \emph{pontas}, $x$ e $z$, somos obrigados a realizar um \emph{loop} sob estes elementos, como no pior caso podemos acabar percorrendo $b-1$ elementos, podemos dizer que esta etapa tem custo $\Oh(b)$. Já para encontrar $y$, iteramos sob os blocos em si, portanto, no pior caso, gastamos $\Oh(d)$ para o seu cálculo. Com isso, temos que a consulta \texttt{find\_sum} tem um custo final $\Oh(\max(b, d))$

Com o intuito de maximizarmos a eficiência desta função, queremos encontrar um tamanho de bloco $b$ ótimo que minimize $\max(b, d)$, isto é, que tornem $b$ e $d$ o mais próximos possível. Para isso, podemos fazer:

\begin{equation}
    b = d \Rightarrow
    b = \frac{n}{b} \Rightarrow
    b^2 = n \Rightarrow
    b = \pm \sqrt{n}
\end{equation}

Portanto, $\sqrt{n}$ é o tamanho ótimo para um bloco, o que implica que a nossa lista sera dividida em $\sqrt{n}$ blocos, dai surge o nome da técnica. Finalmente, temos agora que a consulta \texttt{find\_sum} consome tempo $\Oh(\sqrt{n})$.


%% ------------------------------------------------------------------------- %%
\section{Ideia}
\label{sec:rmsf-ideia}
% mostrar implementação de 1 ou 2 métodos internos



%% ------------------------------------------------------------------------- %%
\section{Consultas Get MSF e Get MST Cost}
\label{sec:rmsf-get-msf}






%% ------------------------------------------------------------------------- %%
\section{Rotina Add Edge}
\label{sec:rmsf-add-edge}



%% ------------------------------------------------------------------------- %%
\section{Complexidade}
\label{sec:rmsf-complexidade}
%   falar da alternativa usada no paper do andre para contornar
%   falar da maneira implemetada, seguindo as linhas sugeridas pelo demaine
%   falar o que falta e citar paper que a cris mandou