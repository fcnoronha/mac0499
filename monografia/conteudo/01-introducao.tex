%!TeX root=../tese.tex
%("dica" para o editor de texto: este arquivo é parte de um documento maior)
% para saber mais: https://tex.stackexchange.com/q/78101/183146

% Escrever bem é uma arte que exige muita técnica e dedicação e,
% consequentemente, há vários bons livros sobre como escrever uma boa
% dissertação ou tese. Um dos trabalhos pioneiros e mais conhecidos nesse
% sentido é o livro de
% Umberto Eco~\cite{eco:09} % usando o estilo alpha
% Umberto~\citet{eco:09} % usando o estilo plainnat
% intitulado \emph{Como se faz uma tese}; é uma leitura bem interessante mas,
% como foi escrito em 1977 e é voltado para trabalhos de graduação na Itália,
% não se aplica tanto a nós.

% Sobre a escrita acadêmica em geral, John Carlis disponibilizou um texto curto
% e interessante~\citep{carlis:09} em que advoga a preparação de um único
% rascunho da tese antes da versão final. Mais importante que isso, no
% entanto, são os vários \textit{insights} dele sobre a escrita acadêmica.
% Dois outros bons livros sobre o tema são \emph{The Craft of Research}~\citep{craftresearch}
% e \emph{The Dissertation Journey}~\citep{dissertjourney}. Além disso, a USP
% tem uma compilação de normas relativas à produção de documentos
% acadêmicos~\citep{usp:guidelines} que pode ser utilizada como referência.

% Para a escrita de textos especificamente sobre Ciência da Computação, o
% livro de Justin Zobel, \emph{Writing for Computer Science}~\citep{zobel:04}
% é uma leitura obrigatória. O livro \emph{Metodologia de Pesquisa para
%   Ciência da Computação} de
% %Raul Sidnei Wazlawick~\cite{waz:09} % usando o estilo alpha
% Raul Sidnei~\citet{waz:09} % usando o estilo plainnat
% também merece uma boa lida. Já para a área de Matemática, dois livros
% recomendados são o de Nicholas Higham, \emph{Handbook of Writing for
%   Mathematical Sciences}~\citep{Higham:98} e o do criador do \TeX{}, Donald
% Knuth, juntamente com Tracy Larrabee e Paul Roberts, \emph{Mathematical
%   Writing}~\citep{Knuth:96}.

% Apresentar os resultados de forma simples, clara e completa é uma tarefa que
% requer inspiração. Nesse sentido, o livro de
% %Edward Tufte~\cite{tufte01:visualDisplay}, % usando o estilo alpha
% Edward~\citet{tufte01:visualDisplay}, % usando o estilo plainnat
% \emph{The Visual Display of Quantitative Information}, serve de ajuda na
% criação de figuras que permitam entender e interpretar dados/resultados de forma
% eficiente.

% Além desse material, também vale muito a pena a leitura do trabalho de
% %Uri Alon \cite{alon09:how}, % usando o estilo alpha
% Uri \citet{alon09:how}, % usando o estilo plainnat
% no qual apresenta-se uma reflexão sobre a utilização da Lei de Pareto para
% tentar definir/escolher problemas para as diferentes fases da vida acadêmica.
% A direção dos novos passos para a continuidade da vida acadêmica deveria ser
% discutida com seu orientador.


%% ------------------------------------------------------------------------- %%
\chapter{Introdução}
\label{cap:introducao}

Estruturas de dados retroativas bla bla bla

%% ------------------------------------------------------------------------- %%
\section{Retroatividade Parcial}
\label{sec:retroatividade_parcial}

%% ------------------------------------------------------------------------- %%
\section{Retroatividade Total}
\label{sec:retroatividade_total}
