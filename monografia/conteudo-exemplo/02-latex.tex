%!TeX root=../tese.tex
%("dica" para o editor de texto: este arquivo é parte de um documento maior)
% para saber mais: https://tex.stackexchange.com/q/78101/183146

\chapter{Usando o \LaTeX{} e este modelo}

Não é necessário que o texto seja redigido usando \LaTeX{}, mas seu
uso é fortemente recomendado, pois ele facilita diversas etapas do
trabalho e o resultado final é muito bom\footnote{O uso de um sistema de
controle de versões, como mercurial (\url{mercurial-scm.org}) ou git
(\url{git-scm.com}), também é altamente recomendado.}. Este modelo é
distribuído com uma ``colinha'' dos principais comandos \LaTeX{} e inclui
comentários explicativos para auxiliá-lo com
ele, sendo composto dos arquivos de exemplo
(\texttt{tese.tex}, \texttt{artigo.tex},
\texttt{apresentacao.tex} e \texttt{poster.tex}) e de
arquivos auxiliares:\looseness=-1

\begin{itemize}
  \item Arquivos com o conteúdo do trabalho:
  \begin{itemize}
    \item \texttt{conteudo/metadados.tex} (título, banca etc.)
    \item \texttt{conteudo/paginas-preliminares.tex} (sumário, dedicatória etc.)
    \item \texttt{conteudo/capitulos.tex}, \texttt{conteudo/apendices.tex},
          \texttt{conteudo/anexos.tex} e demais arquivos carregados por eles
          (\texttt{XX-*.tex}, \texttt{apendice-pseudocodigo.tex}, \texttt{anexo-faq.tex})
    \item \texttt{bibliografia.bib} (dados bibliográficos)
  \end{itemize}

  \item Arquivos com as \textit{packages} usadas e suas configurações (leia
        os comentários neles se quiser modificar algum aspecto do
        documento ou acrescentar alguma \textit{package}):
  \begin{itemize}
    \item \texttt{extras/basics.tex} (\textit{packages} e configurações essenciais),
          \texttt{extras/fonts.tex} (definição das fontes do documento) e
          \texttt{extras/floats.tex} (configurações e melhorias para \textit{floats})
    \item \texttt{extras/imeusp-formatting.tex} (aparência: espaçamento, sumário etc.)
    \item \texttt{extras/index.tex} (configuração do índice remissivo)
    \item \texttt{extras/hyperlinks.tex} (configuração das referências cruzadas)
    \item \texttt{extras/source-code.tex} (exibição de código-fonte e pseudocódigo)
    \item \texttt{extras/utils.tex} (\textit{packages} adicionais diversas)
    \item \texttt{extras/bibconfig.tex} (configuração da bibliografia)
  \end{itemize}

  \item Outros arquivos auxiliares (geralmente não precisam ser editados):
  \begin{itemize}
    \item \texttt{extras/languages.tex} (suporte a múltiplas línguas)
    \item \texttt{extras/imeusp-thesis.tex} (formatação da capa e páginas preliminares)
    \item \texttt{extras/imeusp-headers.sty} (formatação dos cabeçalhos)
    \item \texttt{extras/lstpseudocode.sty} (suporte a pseudocódigo com \textsf{listings})
    \item \texttt{extras/annex.sty} (permite adicionar anexos) e
          \texttt{extras/appendixlabel.sty} (melhora a lista de
          apêndices/anexos no sumário)
    \item \texttt{froufrou.sty} (divisões com ornamentos/florões, como mais abaixo)
    \item \texttt{extras/beamer*.sty} (\textit{layouts} e cores para
          apresentações e \textit{posters})
    \item \texttt{extras/plainnat-ime.*} (estilo plainnat para bibliografias)\index{biblatex}
    \item \texttt{extras/alpha-ime.bst} (estilo alpha para bibliografias com
          bibtex)\index{bibtex}
    \item \texttt{extras/natbib-ime.sty} (tradução da \textit{package}
          padrão natbib)\index{natbib}
    \item \texttt{hyperxindy.xdy} (configuração para xindy) e
          \texttt{mkidxhead.ist} (configuração para makeindex) --- criados automaticamente
    \item \texttt{latexmkrc} e \texttt{Makefile} (automatizam a geração do
          documento com os comandos \textsf{latexmk} e \textsf{make} respectivamente)
  \end{itemize}
\end{itemize}

\froufrou

Para compilar o documento, basta executar o comando \textsf{latexmk} (ou
\textsf{make})\footnote{Você também pode usar \textsf{latexmk poster},
\textsf{make apresentacao} etc.}. Talvez seu editor ofereça uma
opção de menu para compilar o documento, mas ele provavelmente depende do
\textsf{latexmk} para isso. \LaTeX{} gera diversos arquivos auxiliares
durante a compilação que, em algumas raras situações, podem ficar
inconsistentes (causando erros de compilação ou erros no \textsc{pdf} gerado,
como referências faltando ou numeração de páginas incorreta no sumário).
Nesse caso, é só usar o comando \textsf{latexmk -C} (ou \textsf{make clean}),
que apaga todos os arquivos auxiliares gerados, e em seguida rodar
\textsf{latexmk} (ou \textsf{make}) novamente.

\section{Instalação do \LaTeX{}}
\label{sec:install}

\LaTeX{} é, na verdade, um conjunto de programas. Ao invés de procurar e
baixar cada um deles, o mais comum é baixar uma coleção com todos eles juntos.
Há duas coleções desse tipo disponíveis: MiK\TeX{} (\url{miktex.org}) e
\TeX{}Live (\url{www.tug.org/texlive}). Ambos funcionam em Linux, Windows e
MacOS X. Em Linux, \TeX{}Live costuma estar disponível para instalação junto
com os demais opcionais do sistema. Em MacOS X, o mais popular é o Mac\TeX{}
(\url{www.tug.org/mactex/}), a versão do \TeX{}Live para MacOS X. Em Windows,
o mais comumente usado é o MiK\TeX{}.

Por padrão, eles não instalam tudo que está disponível, mas sim apenas os
componentes mais usados, e oferecem um gestor de pacotes que permite adicionar
outros. Embora uma instalação completa do \LaTeX{} seja relativamente grande
(perto de 5GB), em geral vale a pena instalar a maior parte dos componentes.
Se você preferir uma instalação mais ``enxuta'', não deixe de incluir tudo
que é necessário para este modelo, como indicado no arquivo README.md.

Também é muito importante ter o \textsf{latexmk} (ou o \textsf{make}). No
Linux, a instalação é similar à de outros programas. No MacOS X e no Windows,
\textsf{latexmk} pode ser instalado pelo gestor de pacotes do MiK\TeX{} ou
\TeX{}Live. Observe que ele depende da linguagem \textsf{perl}, que precisa
ser instalada à parte no Windows (\url{www.perl.org/get.html}).

\section{Documentação sobre \LaTeX}
\label{sec:docs}

Há muito material sobre \LaTeX{} na Internet, mas também há muita informação
obsoleta (incluindo trechos da própria documentação oficial!). Em particular,
você pode ignorar explicações sobre como converter arquivos no formato
\textsc{dvi} gerados por \LaTeX{} em \textsc{pdf}: As versões atualmente
recomendadas de \LaTeX{} (cf. Seção~\ref{sec:versions}) geram arquivos
\textsc{pdf} diretamente. Quanto a imagens, os formatos de arquivo
\textsc{ps/eps} (PostScript e Encapsulated PostScript) não são adequados
para essas novas versões de \LaTeX{}; elas trabalham com arquivos de imagem
nos formatos \textsc{pdf}, \textsc{png} e \textsc{jpeg}. Finalmente,
recursos gráficos normalmente não usam mais \textit{packages} como
\textsf{pstricks}, \textsf{eepic} ou outras tradicionalmente citadas;
ao invés disso, \textsf{PGF/TikZ} é a ferramenta mais comum.

Como dito anteriormente, \LaTeX{} é, na verdade, um conjunto de programas
e, em geral, instalamos coleções pré-prontas com todos eles. Essas coleções
(\TeX{}Live e MiK\TeX{}) contém também a documentação das \textit{packages}
incluídas: Basta digitar \textsf{texdoc nome-da-package} (\TeX{}Live) ou
\textsf{mthelp nome-da-package} (MiK\TeX{}) para ter acesso à documentação
correspondente. \textsf{texdoc/mthelp} incluem também alguns tutoriais e
textos introdutórios.

Um possível caminho para o aprendizado é começar com o
Capítulo~\ref{chap:tutorial} deste modelo e o conteúdo em
\url{overleaf.com/learn}, que tem escopo similar mas também inclui várias
páginas sobre como utilizar recursos específicos. Após esse contato inicial,
o tutorial em \url{tug.org/twg/mactex/tutorials/ltxprimer-1.0.pdf} é
bastante abrangente e detalhado. Depois que você estiver razoavelmente
familiarizado com a linguagem, utilize o manual de referência que pode ser
acessado em \url{latexref.xyz} ou com \textsf{texdoc latex2e} (disponível
também em francês, com \textsf{texdoc latex2e-fr.pdf}, e em espanhol, com
\textsf{texdoc latex2e-es.pdf}). Para os principais comandos do modo
matemático, veja \textsf{texdoc undergradmath}.

Existem também diversos bons livros sobre \LaTeX{} (embora em geral um
tanto antigos), dos quais destacamos dois:

\begin{enumerate}

  \item A quarta edição de ``A Guide to \LaTeX'', de Helmut Kopka
        e Patrick W. Daly (publicada em 2003), além de uma ótima
        introdução, aborda vários tópicos relativamente avançados
        e úteis\footnote{Uma versão não-final está disponível em
        \url{www2.mps.mpg.de/homes/daly/GTL/gtl_20030512.pdf}.}.
  \item A segunda edição de ``The \LaTeX{} Companion'' (publicada em
        2004) é um livro quase obrigatório, pois discute em detalhes
        praticamente todos os recursos e \textit{packages} importantes
        de \LaTeX{}, servindo tanto para o aprendizado quanto como
        material de referência.

\end{enumerate}

Para dúvidas pontuais, o sítio \url{tex.stackexchange.com} é um
fórum de perguntas e respostas sobre \LaTeX{} muito útil, pois os
principais desenvolvedores do sistema participam das discussões, e o
sítio \url{texfaq.org} é bastante abrangente e atualizado.

\subsection{Outros Recursos}

Existem inúmeras alternativas aos materiais citados acima; outros exemplos de
textos introdutórios são \url{www.maths.tcd.ie/~dwilkins/LaTeXPrimer/GSWLaTeX.pdf}
e \url{www.andy-roberts.net/writing/latex}. Em português, você pode
consultar \url{polignu.org/sites/polignu.org/files/latex/latex-fflch.pdf}
e \url{git.febrace.org.br/material-latex/material-latex} (este precisa ser
baixado e compilado). O canal \url{youtube.com/c/anteroneves} tem vários
vídeos instrutivos em português. \textsf{texdoc/mthelp} incluem ainda
opções como ``The Not So Short Introduction to \LaTeXe{}'' (\textsf{texdoc
lshort-eng}; há uma versão em português, mas não está em dia com o
original) e ``A Simplified Introduction to \LaTeX{}'' (\textsf{texdoc
simplified-intro}). Versões recentes do \LaTeX{} incluem também o
``\LaTeXe{} via exemplos'' (\textsf{texdoc latex-via-exemplos}), em português.

O sítio \url{ctan.org} é o repositório semi-oficial das \textit{packages}
\LaTeX{} e sua documentação; \TeX{}Live e MiK\TeX{} são construídas a
partir do que está nesse site, então a última versão estável de qualquer
\textit{package} (e da documentação acessível com \textsf{texdoc/mthelp})
em geral está ali.

A documentação de referência mais importante sobre os recursos matemáticos
é acessível com \textsf{texdoc amsmath}, \textsf{texdoc amsthm} e
\textsf{texdoc mathtools}; \textsf{texdoc maths-symbols} agrega os símbolos
matemáticos disponíveis. Para uma lista completa de todos os símbolos
disponíveis com \LaTeX{}, use \textsf{texdoc symbols-a4} (esse documento
tem mais de 300 páginas!).

\textsf{texdoc fntguide} explica como funciona a gestão de fontes de
\LaTeX{} (mas note que \LuaLaTeX{} e \XeLaTeX{} usam outro mecanismo;
veja \textsf{texdoc fontspec}). Você pode ver exemplos de fontes
disponíveis para \LaTeX{} em \url{tug.org/FontCatalogue}.

Minúcias sobre o funcionamento interno do sistema estão descritas em
\textsf{texdoc source2e} e, sobre as classes padrão (\textsf{article, book}
etc.), em \textsf{texdoc classes}. Você normalmente não vai usar esses
documentos, mas eles podem servir para esclarecer algum detalhe.
\textsf{texdoc clsguide} é um guia para a criação de novas classes e
\textit{packages}, e \textsf{texdoc macros2e} é uma lista de comandos
especialmente úteis para isso.

Quando você se tornar um usuário avançado, pode se interessar em conhecer
melhor a linguagem \TeX{}, que está na base do \LaTeX{}. ``The \TeX{} book'',
de Donald Knuth (o criador do \TeX), é amplamente recomendado, mas há três
livros completos a respeito que são instalados com \LaTeX{}: ``A gentle
introduction to \TeX{}'' (\textsf{texdoc gentle}), ``\TeX{} for the
impatient'' (\textsf{texdoc impatient}) e ``\TeX{} by topic'' (\textsf{texdoc
texbytopic}).
