% Poster get from https://github.com/victorsenam/tcc/blob/master/poster/main.tex

\documentclass[final]{beamer}
\usepackage[size=a1,orientation=portrait,scale=1.3]{beamerposter}

\usepackage[brazil]{babel}
\usepackage[utf8]{inputenc}
\usepackage[T1]{fontenc}

\usepackage{tikz}
\usetikzlibrary{matrix,shapes,positioning,shadows,trees,patterns}

\usepackage[shortlabels]{enumitem}
\usepackage[numbers]{natbib}
\bibliographystyle{plainnat}

\sloppy

%----------------------------------------------------------------------------------------
%	SHORTCUTS
%----------------------------------------------------------------------------------------
\newcommand{\B}[1]{\mathbb{#1}}
\newcommand{\Cl}[1]{\ensuremath{\mathcal{#1}}}

\newcommand{\sse}{\Leftrightarrow}
\newcommand{\so}{\Rightarrow}
\newcommand{\se}{\Leftarrow}
\newcommand{\rec}{\leftarrow}

\newcommand{\tdots}{\,.\,.\,}

%----------------------------------------------------------------------------------------
%	BEAMER STYLE
%----------------------------------------------------------------------------------------

\usetheme{poster}
\setbeamercolor{block title}{fg=dblue,bg=white}
\setbeamercolor{block body}{fg=black,bg=white}
\setbeamercolor{block alerted title}{fg=dblue,bg=gray!50}
\setbeamercolor{block alerted body}{fg=black,bg=gray!20}
\setbeamercolor{block prob}{fg=black,bg=white}
\setbeamertemplate{caption}[numbered]

%----------------------------------------------------------------------------------------
%	CUSTOM STYLING
%----------------------------------------------------------------------------------------

\newenvironment<>{prob}{
    \begin{beamercolorbox}[sep=1ex,center,dp={1ex}]{block prob}
    \textcolor{dblue}{\textbf{Problema:}}\itshape
}{\end{beamercolorbox}}

\newcommand\halfcol{\column{.46\textwidth}}
\newcommand\onethirdcol{\column{.31\textwidth}}

%----------------------------------------------------------------------------------------
%	POSTER
%----------------------------------------------------------------------------------------

\title{Link-cut trees e aplicações em estruturas de dados retroativas}
\author{Felipe Castro de Noronha \hspace{200pt} Orientadora: Cristina Gomes Fernandes}
\institute{\vspace{10pt}Departamento de Ciência da Computação,
Instituto de Matemática e Estatística, Universidade de São Paulo}


\begin{document}
\begin{frame}[fragile]\centering
  \vspace{-.5\baselineskip}
  \begin{columns}[T]

    % ----------------------------------------------------------------------------------------
    % PRIMEIRA COLUNA
    % ----------------------------------------------------------------------------------------
    \onethirdcol
    \begin{alertblock}{Resumo}
      Estruturas de dados retroativas permitem a realização de operações que afetam não somente o estado atual da estrutura, mas também qualquer um de seus estados passados. Além disso, uma link-cut tree é uma estrutura de dados que permite a manutenção de uma floresta de árvores enraizadas com peso nas arestas, e onde os nós de cada árvore possuem um número arbitrário de filhos. Tal estrutura é muito utilizada como base para o desenvolvimento de estruturas de dados retroativas, e neste trabalho estudaremos as versões retroativas dos problemas de union-find e floresta geradora mínima.  Para isso, implementamos essas estruturas em \texttt{C++} e descrevemos as ideias por trás de seus funcionamentos. Ademais, apresentamos uma melhoria da solução originalmente apresentada para a floresta geradora mínima retroativa, que retira limitações sem piorar sua performance.
    \end{alertblock}

    \begin{block}{bla}
      blabla blabla blabla blabla blabla blabla blabla blabla blabla blabla blabla blabla blabla blabla blabla blabla blabla blabla blabla blabla blabla blabla blabla blabla blabla blabla blabla blabla blabla blabla blabla blabla blabla blabla blabla blabla blabla blabla blabla blabla blabla blabla blabla blabla blabla blabla blabla blabla blabla blabla blabla blabla blabla blabla
    \end{block}
    \begin{block}{bla}
      blabla blabla blabla blabla blabla blabla blabla blabla blabla blabla blabla blabla blabla blabla blabla blabla blabla blabla blabla blabla blabla blabla blabla blabla blabla blabla blabla blabla blabla blabla blabla blabla blabla blabla blabla blabla blabla blabla blabla blabla blabla blabla blabla blabla blabla blabla blabla blabla blabla blabla blabla blabla blabla blabla
    \end{block}
    \begin{block}{bla}
      blabla blabla blabla blabla blabla blabla blabla blabla blabla blabla blabla blabla blabla blabla blabla blabla blabla blabla blabla blabla blabla blabla blabla blabla blabla blabla blabla blabla blabla blabla blabla blabla blabla blabla blabla blabla blabla blabla blabla blabla blabla blabla blabla blabla blabla blabla blabla blabla blabla blabla blabla blabla blabla blabla
    \end{block}
    \begin{block}{bla}
      blabla blabla blabla blabla blabla blabla blabla blabla blabla blabla blabla blabla blabla blabla blabla blabla blabla blabla blabla blabla blabla blabla blabla blabla blabla blabla blabla blabla blabla blabla blabla blabla blabla blabla blabla blabla blabla blabla blabla blabla blabla blabla blabla blabla blabla blabla blabla blabla blabla blabla blabla blabla blabla blabla
    \end{block}

    \begin{block}{bla}
      blabla blabla blabla blabla blabla blabla blabla blabla blabla blabla blabla blabla blabla blabla blabla blabla blabla blabla blabla blabla blabla blabla blabla blabla blabla blabla blabla blabla blabla blabla blabla blabla blabla blabla blabla blabla blabla blabla blabla blabla blabla blabla blabla blabla blabla blabla blabla blabla blabla blabla blabla blabla blabla blabla
    \end{block}




    % ----------------------------------------------------------------------------------------
    % SEGUNDA COLUNA
    % ----------------------------------------------------------------------------------------
    \onethirdcol

    \begin{block}{bla}
      blabla blabla blabla blabla blabla blabla blabla blabla blabla blabla blabla blabla blabla blabla blabla blabla blabla blabla blabla blabla blabla blabla blabla blabla blabla blabla blabla blabla blabla blabla blabla blabla blabla blabla blabla blabla blabla blabla blabla blabla blabla blabla blabla blabla blabla blabla blabla blabla blabla blabla blabla blabla blabla blabla
    \end{block}



    % ----------------------------------------------------------------------------------------
    % TERCEIRA COLUNA
    % ----------------------------------------------------------------------------------------
    \onethirdcol
    \begin{block}{bla}
      blabla blabla blabla blabla blabla blabla blabla blabla blabla blabla blabla blabla blabla blabla blabla blabla blabla blabla blabla blabla blabla blabla blabla blabla blabla blabla blabla blabla blabla blabla blabla blabla blabla blabla blabla blabla blabla blabla blabla blabla blabla blabla blabla blabla blabla blabla blabla blabla blabla blabla blabla blabla blabla blabla
    \end{block}

    \begin{alertblock}{Resumo}
      Estruturas de dados retroativas permitem a realização de operações que afetam não somente o estado atual da estrutura, mas também qualquer um de seus estados passados. Além disso, uma link-cut tree é uma estrutura de dados que permite a manutenção de uma floresta de árvores enraizadas com peso nas arestas, e onde os nós de cada árvore possuem um número arbitrário de filhos. Tal estrutura é muito utilizada como base para o desenvolvimento de estruturas de dados retroativas, e neste trabalho estudaremos as versões retroativas dos problemas de union-find e floresta geradora mínima.  Para isso, implementamos essas estruturas em \texttt{C++} e descrevemos as ideias por trás de seus funcionamentos. Ademais, apresentamos uma melhoria da solução originalmente apresentada para a floresta geradora mínima retroativa, que retira limitações sem piorar sua performance.
    \end{alertblock}


    \begin{block}{Informações e contato}
      Para mais informações, acesse a página do trabalho: \textcolor{jblue}{{\url{https://linux.ime.usp.br/~felipen/mac0499/}}}
      \vskip2ex
      Endereço para contato: \textcolor{jblue}{{\url{felipe.castro.noronha@usp.br}}}

    \end{block}

    \begin{block}{Referências}
      \scriptsize{\begin{thebibliography}{99}
          \bibitem{aho}
          Aho, Alfred V. and Corasick, Margaret J.,
          ``\textbf{Efficient string matching: an aid to bibliographic search},'' in \textit{Commun. ACM}, 1975, 18 (6), pp. 333–340.

          \bibitem{weiner}
          Peter Weiner,
          ``\textbf{Linear pattern matching algorithms},'' in \textit{14th Symposium on Switching and Automata Theory}, 1973.

          \bibitem{ukkonen}
          Ukkonen, Esko,
          ``\textbf{On-line construction of suffix trees}'', in \textit{Algorithmica}, 1995, 14 (3), pp. 249–260.
        \end{thebibliography}}
    \end{block}


  \end{columns}
\end{frame}
\end{document}

